
\documentclass[a4paper,reqno,oneside]{article}
\pdfoutput=1
\include{mathcommands.extratex}
\begin{document}
\title{On The Monoindal Endofunctor In Sparse Categories}
\author{Max Vazquez}
\maketitle


\begin{abstract}
    This paper proves that the monoindal endofunctor on the category of topological complexes in an (infinite) finite dense category is a bijection, and gives an explicit method for computing the action on a topological complex by computing all its components.  We study the action on the action of a single object with any subcategory of $\C$.  Moreover, we compute the action on the action of a set of objects in an infinite dense category.
\end{abstract}

\section*{Introduction}

One of the most important concepts in algebraic geometry is to study the homotopy-completion of a complex.  For example, a diagram may be given as follows:
$$
\begin{tikzcd}
	x & \ar{rr}{\phi}& y \\
	& z
\end{tikzcd}
$$
where $x$ and $y$ are points, $\phi$ is some field element, and $z$ is a point in $C$, or an arbitrary point in $C$.  If we were given a section of $x$ by a line $l: C/Z \to C$, then we can consider what happens if we apply the same method to consider the complex of continuous maps from $X$ to $Y$ given by the map $Z \times X \times Y \to Z \times Y$ acting on $l$.  A continuous map may be thought of as sending each coordinate of the input to a corresponding coordinate of the output; this may be more directly mapped into a continuous map which acts on $l$ by choosing a new coordinate $v$ on $l(z)$ such that the diagram becomes
$$
\begin{tikzcd}
	x & \ar{r}{\phi}& v \\
	y & \ar{rr}{\phi}& z
\end{tikzcd}
$$
This completes our discussion of the section of $x$ by the line $l$.
The complex $C$ has no global sections, and therefore the section must satisfy two conditions: first, it must contain an open path in order for it to form a non-empty chain complex; second, it must have a fixed point in order for it to be a discrete complex.  

But if $C$ contains an open path, the fact that $C$ is a discrete complex implies there are multiple non-empty chains $H_n$ which intersect and form a non-empty chain complex; hence, there will be at least one local section $H_{i+j}$ in $C$.  These sections may themselves contain other open paths, and these paths may form cycles in the complex.  An example is the Hamiltonian cycle complex, whose cycles are closed paths.   In the context of topology we would want to understand that the Hamiltonian cycle complex has a global section which contains both $A^3$ and $B^3$, but not necessarily a local section.  That is, if we had two distinct closed paths $L \to N$, then they should form a unique closed path between them.  But this isn't the case with the Hamiltonian cycle complex since there is another non-unique closed path $C \to B$ which is contained in $C$ as well. 

Another way to think of this problem is to say that we want a non-trivial solution when a complex is generated by locally finite summands.  Then, instead of trying to understand why the summands $L$ and $N$ do not intersect, we ask whether they intersect in some way.  We get this answer only if $C$ is closed in the sense that either of the endpoints of a closed path must be in a local section of $C$.  Furthermore, since the topological cycle complex is generated by local sections, any closed path between two local sections should be the same as a closed path between their intersection.   

In order to prove this result, we need to show that the topological cycle complex has a global section.  Thus, we must define the category $\Topo(\C)$ of topological complexes in an arbitrary category $\C$.  Every class of topological complexes in $\C$ has a canonical universal property: every closed chain complex with its terminal object must be self-contained in a global section.  However, we want to verify a number of facts about the topological cycle complex.  First, we require that a topological complex contains no loops.  Next, we require that an element of the complex is neither a single point nor a line segment.  Finally, we require that a topological complex is a discrete complex.  To see that topological complexes are discrete, we first need to prove that the category $\Topo(\C)$ is finitely complete and colimit complete.  

We begin by defining our main data types.  There is an infinite series $C$ of objects in $\C$ together with a morphism $f: C \to D$ between those objects and a function $f: C \rightarrow D$ called \textit{splitting}.  Splitting means we call it a \textit{splitting pair}, since we only work with splittings.  Each splitting pair consists of a non-empty subset $V = [f]$ of $\C$, called \textit{vertices}, and an object $a \in V$ called \textit{boundary} and $b \in V$ called \textit{inside}.  Since the boundary is always the smallest vertex, we use the convention that $a = b$.  More generally, a splitting pair consists of three things: 
\begin{itemize}
\item An object $A \in V$, called the \textit{domain},
\item A collection of edges $\left\{e_i: A \rightarrow A^{i+1}\right\}_{i \in I}$, where $I$ is a set of integers,
and
\item A morphism $p : A \rightarrow B$, called \textit{retract}, such that 
$$
[f] \twoheadrightarrow A^{i+1} \quad\text{for all } 0 \leq i \leq j < j + 1
$$
is a retraction of $e_j$ for some $0 \leq i \leq j - 1$.
\end{itemize}
\begin{definition}
    A collection of splitting pairs is said to be \textit{finite}, if all the finite subsets $V = [f]$ and $I$ are finite.
\end{definition}
Next, we define the \textit{endomorphism ring}:
$$
R_A := \{ e: A \rightarrow A^{i+1}\}_{\in I}.
$$
As a reminder, the \textit{vector space} of a splitting pair $(V, V_{a})$ is the following quotient of the category of objects over $A$:
$$
\Hom_\C(A, A^{i+1}):= \Hom_{\Set}(A, A^{i+1}).
$$
These objects are said to be \textit{colimit-preserving} if the quotient is fully faithful, and thus $\Hom_{\Set}(A, A^{i+1}) = \Hom_{\Set}(A, A^{i+1})$.
To see that a cochain complex is a colimit-preserving category, let us take an example where $V = \mathbb{C}$.  One can check that it satisfies all the assumptions required for a chain complex.  Recall that a chain complex consists of several objects, and for each $V \in \C$, the chain complex consisting of the objects in $V$ is denoted by $V\text{-}\mathrm{Ch}$.  Thus, the chain complex $(V\text{-}\mathrm{Ch})(V)$ consists of a set of vertices $V \in \C$ together with two functions $V \rightarrow V^{2}$ and $V^{2} \rightarrow V$ called \textit{projections} and \textit{composites}.  The projections and composites determine two projections $U \rightarrow V$ and $W \rightarrow V^{2}$, respectively:
$$
U:=\{v \in V \mid \exists w \in V^{2} \text{ such that } V \hookrightarrow Wv \neq W^{-1}w\}.
$$
The chain complex $(V\text{-}\mathrm{Ch})(V)$ also consists of a set of edges $\{E_i: V_i \rightarrow V_{i+1}\}_{i \in I}$ where $I$ is a set of integers, and a function $V \rightarrow V^{i+1}$ called \textit{decomposition} such that $V \rightarrow U \circ V$ is a decomposition map for some $U \in R_{A}^{i+1}$.  More precisely, the decomposed elements in a splitting pair $(V, E_i)$ are precisely those elements of $V$ and $E_i$ which are contained in a non-empty loop in $E_i$, for all $i \in I$.  A splitting pair $(V, E_i)$ is said to be \textit{closed} if $(V, E_i)$ consists of a vertex $V \in \C$ together with a decomposition map $F: V \rightarrow V^{i+1}$, such that each of the decomposed elements of $(V, E_i)$ is contained in $F(V)$.

With all this information, we can now build a definition of a topoid.
\begin{definition}[Definition~9.16 of \cite{johnstone2007topological}]
    A topological complex $\C$ is a \textit{topoid} if it is a cochain complex and is colimit-preserving.
\end{definition}
Using our definitions, we can now establish a notion of being able to describe $\C$ using endomorphisms.  Note that the endomorphism ring $R_A$ consists of all the elements $e: A \rightarrow A^{i+1}$, and thus an endomorphism ring does not require us to explicitly enumerate the different endomorphisms of $\C$.  We make the following observation:
\begin{lemma}
    Let $f: C \to D$ be a morphism of splittings. Then the following holds for the totalization of the topological complex:
    $$
    \Total_{[f]}(C) = \hom_{\C}(D, D^{i+1}).
    $$
    The action of an endomorphism ring on a split pair $A = (A, A_{b}, e_b, A_{a}, e_a)$ is given by the composite endomorphism $A_{b}: A \rightarrow A^{i+1}$ followed by the composition endomorphism $A_{a}: A \rightarrow A^{i+1}$ which intertwines with the action of $A$ on $e_b$ and $e_a$.
\end{lemma}

We can now characterize how many different endomorphisms exist on an infinite topology:
$$
|R_A|_{[f]} = \left\lvert \Hom_{\C}(D, D^{i+1}) \right\rvert.
$$
Therefore, we have a bijection between the totalization of an infinite topology and the number of endomorphisms on an infinite topology:
\begin{corollary}[Corollary~8.4.31 of \cite{johnstone2007topological}]
    Let $f: C \to D$ be a morphism of finite splits. Then there is an inverse bijection between the totalization of an infinite topology and the number of endomorphisms on an infinite topology.  In particular, there is an inverse bijection between the totalization of a finite topology and the number of endomorphisms in the ring $R_A$.
\end{corollary}
Moreover, this bijection may be made even more precise by including any endomorphism ring as a subobject of $R_A$.

For a full list of endomorphism rings, we refer the reader to \cite{johnstone2007topological}.  It is worth noting that $\Set$ is equivalent to $R_A$ by setting $e_a = 1$.  This was stated in \cite{johnstone2007topological}.  A similar argument can be used to prove Corollary~8.4.31 in $\Set$ using the action of $A$ on all the elements of the $e_i$ and all the ones appearing in the decomposition map $F$.  In terms of categories we might try to compute the totalization of $\Set$ without taking the action of an endomorphism ring as a subobject of $R_A$, as this would give a different result than simply using the ring.  However, such a calculation would provide no useful insight, as we did with $\Topo(\Set)$.   Therefore, in this paper, we use the $R_A$ as the only non-subobject of $R_A$ so that the bijection becomes more natural.

Now that we know how to interpret the action of endomorphisms on an infinite topology, we turn to the following question:  How much extra information do we gain by taking endomorphisms instead of the endomorphism rings?  

\section{Methods to Compute the Action on Topological Complexes}

The following result is an application of Kolmogorov's Hopf theory~\cite{kolmogorov2004hopf}.

\begin{theorem}
    Let $\C$ be a finite topological complex. If there exists a nonzero action on $\C$ defined by a nonzero element $A \in R_A$, then the action of $R_A$ on the topological complex $\C$ is given by the product of endomorphisms.
\end{theorem}

Let $S,T$ be finite sets of subsets of $A$, and suppose that there exists a morphism $e: S \to T$ such that $S \subseteq e$ and $T \subseteq e$.  Let $V$ and $W$ be finite vectors which are a subsum of $e(V)$.  We want to compute the following properties of the product of endomorphisms:
\begin{enumerate}
\item[(i)]
The totalization of $\C$ is given by the following:
$$
\Total_{[e]}(C) = \prod_{i=1}^n \hom_{\C}(V_i, W_i).
$$
\item[(ii)]
An endomorphism of $R_A$ on an infinite topology is given by an element $A \in R_A$ which acts on the morphisms
\begin{align*}
    e_k &:= v \mapsto \id_k, \qquad \text{and} \qquad e^{V_i} &= 1_{W_i}, \qquad  \text{with} \qquad e^{W_i} &= (V_i)^* \cdot e^{V_i}.
\end{align*}
\end{enumerate}
We call the resulting product an \textit{action} of $R_A$ on the complex $\C$.

However, given an element $A \in R_A$, we often find the action of $A$ on $\C$ in terms of components, or \textit{components} of $A$.  For instance, we might ask to compute the action of $A$ on the product of objects $\Hom_{\C}(B, B^2)$.  However, $B$ is not a component, because $B$ could be a direct sum of two elements which aren't components.  To correct this situation, we define a \textit{restriction} on $R_A$.  Restricting an endomorphism $A \in R_A$ on the complex $\C$ into an element $A' \in R_{A'}$, we say that $A'$ is a \textit{restriction of $A$} of $\C$ to a restriction of $A$.  

Similarly, let $A \in R_{A'}$ and let $e: S' \to T$ be a morphism of infinite topologies.  Given an infinite complex $\C$, we obtain a morphism $f: V_i \to W_i$ called \textit{pullback} from $A$ to $A'$.  Similarly, given an infinite complex $\C$, we obtain a morphism $g: W_i \to V_i$ called \textit{pushforward} from $A'$ to $A$.  These morphisms are equivalent to the actions of $A'$ on $\C$ as elements of $R_A$.   We will denote this new element $A'$ by $A''$ and let $A'' = A' \times e$ and write $S'' = S' \times e$ for the underlying topological complex of $A'$, so that $S'' = T$.  To compute $A''$ as a component of $R_A$ we compute a restriction of $A''$ along $e$.

We can also express $A''$ in terms of components of $A''$ with an additional restriction of $A'$.  In this case, $A''$ is not a restriction of $A$ to an element of $R_{A'}$, but rather a component of $A'$.  Indeed, the image of $A''$ in the topological complex $\C$ is the same as the pullback $A'' \times e$ from $A''$ to $A'$.

\subsection{Action of $R_A$ on a Topological Complex}

The following result shows that the action of $R_A$ on an infinite topology is simply computed by restricting the action of $A$ on $\C$ onto components of $A$.

\begin{lemma}
    Let $A \in R_A$ be a morphism of infinite topologies. 
    Suppose that $A$ is given as the action of a nonzero element $A' \in R_A$.  Then $A'$ acts by the composite
$$
A'' \twoheadrightarrow A' \times e.
$$
If $A'$ acts by the composite $A'' \twoheadrightarrow A' \times e$ for an endomorphism $e: S'' \to T$, then $A''$ acts by the action of $A'$ on $\C$ as an element of $R_A$.
\end{lemma}

Assume that $A$ acts by the composite $A'' \twoheadrightarrow A' \times e$ and let us fix some elements $S,T$ of $A'$.  For any $v,w \in S$ such that $v \neq w$, we can express $v \times A''$ as a pullback $v \times A'' \times e$.  We then have a morphism $A'' \times e \to A' \times e$ as well as a morphism $e \to A' \times e$.  This provides us with a morphism from $A''$ to $A'$.  Note that $A'$ acts by the composition 
$$
A'' \twoheadrightarrow A' \times e.
$$
as well as $A'' \twoheadrightarrow A' \times e$ which induces a morphism $A'' \times e \to A' \times e$.  We may choose to further restrict $A''$ to $A'$ by setting 
$$
e' \mapsto e' + A' \times e,
$$
so that we have an element $A'''$ of $R_A$ which acts as an element of $R_A$.  This element acts by the composite
$$
A'' \twoheadrightarrow A' \times e + A' \times e'.
$$
Note that we may choose to include the zero element as a component of $A''$.

Given an action of $R_A$ on $\C$ with elements $A \in R_A$, let $A''$ be the result of pulling back $A$ on both sides.  Notice that the restriction of $A''$ to $A'$ is again a restriction of $A$ to $A''$.  Consider an element $c \in A''$ and a morphism $f: A'' \to T$ such that $f(A'') \notin S'$.  Then $c \notin e'$ since $f$ is in bijection with the image of $A''$ in $\C$ of $S''$.  If $f$ is a pushforward, then $f'(A'') \in S'$.  Since $A''$ is a component of $R_A$, we may assume that $f'(A'') \neq c$; then $c$ is the limit of $e'$ and $f'$ vanishes.  Also, $c \in e'$.  Hence, $c \in A$ for some element $c \in A'$.  Now, since $A'$ acts by the composite $A'' \twoheadrightarrow A' \times e + A' \times e',$ then $c \in A$ since the product of endomorphisms agrees with the action of $A'$.  The following assertion shows that $A''$ acts as an element of $R_A$.

\begin{theorem}[Theorem~9.19 of \cite{johnstone2007topological}]
    If $A \in R_A$, then the action of $R_A$ on an infinite topology is computed by restricting the action of $A$ on $\C$ onto components of $A$.
\end{theorem}

Now we turn to the simplest possible example of the action of $R_A$ on an infinite topology, namely the action of a singular object $A \in R_A$.  We first recall some basic facts about endomorphisms and restrictions of $R_A$.

\begin{definition}
    Suppose that $A$ acts by the composite $A'' \twoheadrightarrow A' \times e$. Then we define a \textit{restriction} of $R_A$ which assigns $A''$ to the composite
    $$
    A'' \twoheadrightarrow A''' \twoheadrightarrow A' \times e.
    $$
    Recall that a morphism $A'' \to A'''$ is in bijection with the image of $A''$ in $\C$ of $S''$.
\end{definition}

By virtue of Lemma~\ref{lemma19}, $A''' \in R_A$.  Using the notation and the description above, we can identify the restriction of $A''$ to $A'''$ as the restriction $A' \times e$.  That is, $A''$ acts as an element of $R_A$ as well as as an element of $R_{A'}$; the latter is given by a pullback $A'' \times e$ and a pushforward $A''' \times e$.

\subsection{Action of $R_A$ on an Infinite Complex}

Let us now consider the following example of the action of $R_A$ on an infinite topology: the action of a single object $A \in R_A$.  The element $A$ acts by the action of $A \times e$ on $\C$ with elements $e_i = A_{a_i} \times e$ for all $i \in I$.  Since $e_1 = A_{a_1}$ and $e_2 = A_{a_2}$ for $a_1,a_2 \in A$, we obtain an action of $R_A$ on $\C$ as $A$.  

Consider a morphism $f: V \to W$ such that $f(A') \notin S$.  We can express $f(A')$ as a pullback $f(A') \times e$.  However, the image of $f(A')$ in $\C$ of $S$ is $V$, and so $f'(A')$ is the product $e$ of endomorphisms with respect to $e$.  Therefore, $f'(A') \in S$.

To prove the theorem, we compute $A \times e$ as an element of $R_A$.  By induction on $f(A')$.  If $f(A') = S$ then we set $e$ to be the zero element of $R_A$.  Otherwise, if $f(A') = V$ then we set $e$ to be $v$.  Since $e_1 = A_{a_1}$ and $e_2 = A_{a_2}$ for $a_1,a_2 \in A$, we obtain the following formula:
$$
e = e_1 + A_{a_1}e_2,
$$
which states that $e$ acts by $A \times e$ as well as the zero element of $R_A$.  Inductively, if we have $f(A') = V_i$ for some $i \in I$, we compute $f'(A')$ as follows:
\begin{equation}
    \label{eq:pullback}
    f'(A') = \underbrace{e_i}_{\text{for every } i \in I} + A'_{a_i}e_i.
\end{equation}
Since $f(A') = V$, we get the element
$$
A'' + A'_{a_1}e_2 = A'_{a_1} + A'_{a_2}e_1,
$$
and since $A'$ acts by the composite $A'' \twoheadrightarrow A' \times e + A' \times e'$ which induces a morphism $A'' \times e \to A' \times e$ we obtain
$$
A'' + A'_{a_1}e_2 + A'_{a_2}e_1 = A'_{a_1} + A'_{a_2}e'.
$$
Thus, the $e_i$ are the $e$ on the left hand side of the equal sign.

Now we want to compute $A''$ as a component of $R_A$.  Since $A''$ acts by the composite $A'' \twoheadrightarrow A' \times e + A' \times e$', we compute the elements $e'_i$ via the following formula:
$$
e'_i = e_i - A'_{a_i}e_i,
$$
where we have chosen to omit $e'_0 = e$.  Therefore, by induction on the number of $e'_i$.  If $f(A') = V_i$ for some $i \in I$, we compute $e'_i$ as follows:
\begin{equation}
    \label{eq:pullback1}
    e'_i = v_i + A'_{a_i}e_i.
\end{equation}
If $f(A') = W_j$ for some $i \in I$, we compute $e'_i$ as follows:
\begin{equation}
    \label{eq:pullback2}
    e'_i = A_{a_i}e_j + A'_{a_i}v_i.
\end{equation}
Notice that $e'_i = e_i - A'_{a_i}e_i$ for some $i \in I$.  So $e'_i = e_i - A'_{a_i}e_i + A'_{a_i}e_j$, which is an element of $R_A$.  Note that $e'_i$ acts by the $A'_{a_i}$-morphism $A'_{a_i} \times e \to A'_{a_i} + A'_{a_i}e$, as well as the $A'_{a_i}$-morphism $A'_{a_i}e \to A'_{a_i} + A'_{a_i}e'$.  Since $e'_0 = e$, $e'_1 = e + A'_{a_1}e$, and $e'_2 = e + A'_{a_2}e$, we now have $e_1 = A'_{a_1}e$ and $e_2 = A'_{a_2}e$.  The final step is to choose the element of $R_A$ which acts by the $A'_{a_i}$-morphism $A'_{a_i}e \to A'_{a_i} + A'_{a_i}e'$ as described above.  The remaining elements of $R_A$ can be calculated by doing the same calculation for the morphisms $e'_i$ from Theorem~\ref{thm19}, and multiplying by $A$ and $A'$ respectively.  In essence, $e'_i$ acts by the $A'_{a_i}$-morphism $A'_{a_i}e \to A'_{a_i} + A'_{a_i}e'$ as well as the $A'_{a_i}$-morphism $A'_{a_i}e' \to A'_{a_i} + A'_{a_i}e$.  

We remark that in Theorem~\ref{thm19} we mentioned that $e_i$ are the $e$ on the left hand side of the equal sign.  However, $A''$ acts by the composite $A'' \twoheadrightarrow A' \times e + A' \times e'$ and thus we cannot simply omit $e_0$ when determining $e_i$ from $e'_i$.  As a consequence of this problem, we cannot perform our computations as before, because $R_A$ is a restricted $R_A$-action of $\C$, not an $R_A$-action of $\C$ itself.  In addition, we do not have access to the action of $A$ on $\C$.

\section{A Bijective Method to Computing the Action of a Topological Complex}

With this background information in hand, we can begin on our discussion of the action of $R_A$ on an infinite topology.  The first thing we would like to investigate is how to use this in our computation.  The following is a technical result, but it is convenient to understand the proof in advance.

\begin{proposition}[Theorem~9.22 of \cite{johnstone2007topological}]
    Let $A \in R_A$ be a morphism of infinite topologies. If $A$ acts by the composite $A'' \twoheadrightarrow A' \times e$ for an endomorphism $e: S'' \to T$, then $A''$ acts by the action of $A$ on $\C$ as an element of $R_A$.
\end{proposition}

Let $A''$ be the result of pulling back $A$ on both sides of Theorem~\ref{thm19}.  Consider a morphism $f: A'' \to T$ and let $v \in S''$.  Suppose that $v \in A''$.  Note that $v$ is neither an element of $S''$ nor an element of $A'$.  However, $v$ is in bijection with $A$ in $R_A$.  Therefore, by Proposition~\ref{prop22}, $A''$ acts by $A \times e$ and has an element of $R_A$.

Now that we know how to compute $A''$ as a component of $R_A$, we turn to answering our original question.  Suppose that $v$ is the pullback $f(A'') \times e$ in Lemma~\ref{lemma19}.  Then $v \in S''$ and hence $v$ acts by $A'' \times e$ as an element of $R_A$.  In other words, since $f$ is in bijection with $A''$, we can express $v$ as a pullback $v \times A'' \times e$ in $R_A$.

Next, we will prove an analogous statement for pushing forward $A''$ to $A$.

\begin{definition}[Definition~9.24 of \cite{johnstone2007topological}]
    Let $A''$ be the result of pulling back $A$ on both sides of Theorem~\ref{thm19}. Suppose that $A''$ acts by the composite $A'' \twoheadrightarrow A' \times e$ for an endomorphism $e: S'' \to T$.
    Then we define a \textit{restriction} of $R_A$ which assigns $A''$ to the composite
    $$
    A'' \twoheadrightarrow A''' \twoheadrightarrow A' \times e.
    $$
    Recall that a morphism $A'' \to A'''$ is in bijection with the image of $A''$ in $\C$ of $S''$.
\end{definition}

Using the notation and the description above, we can identify the restriction of $A''$ to $A'''$ as the restriction $A' \times e$.  That is, $A''$ acts by the composite $A'' \twoheadrightarrow A'
\end{document}
 \times e$ and has an element of $R_A$.

Since $A''$ acts by the composite $A'' \twoheadrightarrow A' \times e$ and has an element of $R_A$, we may choose to further restrict $A''$ to $A''$ by setting 
$$
e' \mapsto e' + A' \times e,
$$
so that we have an element $A'''$ of $R_A$ which acts as an element of $R_A$.  This element acts by the composite
$$
A'' \twoheadrightarrow A' \times e + A' \times e'.
$$
Note that we may choose to include the zero element as a component of $A''$.

Again, in this case, we can use the notation and the description above to express $A''$ as a component of $R_A$ along $e$.  Given an action of $R_A$ on $\C$ with elements $A \in R_A$, let $A''$ be the result of pushing forward $A$ along $e$.  Consider a morphism $g: A'' \to V$ such