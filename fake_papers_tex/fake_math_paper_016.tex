
\documentclass[a4paper,reqno,oneside]{article}
\pdfoutput=1
\include{mathcommands.extratex}
\begin{document}
\title{Co-Algebras: Why Do They Exist?}
\author{Max Vazquez}
\maketitle


In this paper, we review some of the fundamental ideas in coalgebras and introduce the notion of a \textit{coalgebra}. 

Our first step will be to understand what coalgebras are (and why they exist). This will help us to study their cohomology abelianization: the concept of a coalgebra and its representation theory. We also explore some of the properties of such algebras that will be useful for the following paper.

\section{Why Does Coalgebra Exists?}

There are many algebras that can be used to construct higher categories of categories (see~\cite[Example 5.8]{AbramskyLurieFreedBradyShulman}) with one exception: coalgebras. These are in fact precisely algebras that can be defined as algebras which have a left adjoint. We describe them in more detail below.

Let $A$ be an algebraic structure on $\mathcal{C}$. A coalgebra over $A$ is an algebra that is both right and left adjoint to $A$. To describe coalgebras in general, it is crucial to understand their homotopy abelianizations. Let $\mathcal{C}$ denote the category of small categories, $\mathsf{Cat}$ being the category of categories.

Suppose that $A$ is the category of sets and $B$ is the category of finite products. The categories of \textit{small categories} ($\mathcal{S}$) have a binary product given by the intersection of two objects, $\cap$: 
$$A \times B \coloneqq \{((x,y),z)\mid x \subseteq y \text{ and } z \in B \}$$ 
where $(x,y)$ denotes a pair of elements. It is then easy to see that $\mathsf{Cat}$ has a left adjoint to $\mathcal{S}$, denoted by $\iota: \mathcal{S} \to \mathcal{S}$, where $A \xrightarrow{\iota} B$. For example, let $A := \mathsf{Set}_{\{\langle \varphi, \psi \rangle : \varphi = \chi\}}$ and let $\iota = \overline{\circ}: \mathsf{Set}_{\{\langle \varphi, \chi \rangle : \varphi = \chi\}} \to \mathsf{Set}_{\{\langle \chi, \varphi \rangle : \varphi = \chi\}}$. Note that we are using notation similar to that of~\cite{AbramskyLurieFreedBradyShulman}, in that we write $A \xrightarrow{\overline{\circ}} B$.

We note that any coalgebra over $A$ must satisfy the following property: If a collection of objects is isomorphic to another collection of objects, then so is its homotopy abelianization. Consider any coalgebra over $A$, say $R = R_{ab} \colon X \otimes Y \to \mathcal{C}$ where $X,Y$ are categories, the objects of $X$ form a collection of objects and the objects of $Y$ form a collection of objects. We denote the objects of $R_{ab}$ by $Z_a, Z_b$, respectively. For each $z \in Z_a$ there exists an object $x \in X$ such that $\cap(x,z) = \emptyset$. We define the homotopy abelianization of $R_{ab}$ as the collection of objects given by the coequalizers of the natural isomorphisms: 
$$\coeq_R(x,z): \bigcup_{x \in X, y \in Y} x \otimes y \to y.$$
Recall that for any collection $A$ and any map $f: X \to Y$ there exists a bijection between the codimension $n$-tuple of $A$ and the $n$-tuple of maps $A \otimes X \to A \otimes Y$. We would like to see if we could define an appropriate collection of objects satisfying the above property. To do so, it would be convenient to look at the corresponding collection of maps $g_0(z) : X_0 \to X_1 \otimes Y$ for every $z \in Z_a$, for $n = 1$ or $2$. In particular, we need to show that the collection of maps $g_0: X_0 \to X_1 \otimes Y$ is finitely generated by all of the maps $f_{ab}: X \to \mathcal{C}$. We define these maps as follows: 
\begin{align*}
g_0(z) &:= z \\
g_0(z) &= g_{ab}(z)
&&&if $x \in \cap(x,z)$ then $g_0(x) \in \cap(x,g_{ab}(z))$ \\
&&&if $x \in \cap(x,z)$ then $g_{ab}(x) \not \in \cap(g_0(x),z)$ \\
&&&if $y \in Y$ then $g_{ab}(y) = 0$
\end{align*}
As you might imagine, $g_0(z) = z$ for every $z \in Z_a$, except when $g_{ab}(x) = 0$. It therefore follows that $g_0: X_0 \to X_1 \otimes Y$ is finitely generated by all of the maps $f_{ab}$.

Since $g_0$ is a bijection, it means that every $g_0(x) \in \cap(x,g_{ab}(z))$ for every $z \in Z_a$ (for any $n \leqslant 1$). This suggests that there is indeed an element $x \in X$ such that $\cap(x,g_0(z)) = \emptyset$. So clearly $f_{ab}: X \to \mathcal{C}$ is the left adjoint of $g_0$, and we conclude that there is a coalgebra $R = R_{ab}$ over $A$ that satisfies the property of having a coequalizer. This explains why coalgebras are so important in mathematics, since we hope to apply them in our work.

\subsection{The Homological Abelianization of a Coalgebra}

When we talk about a coalgebra over an algebraic structure $A$ we usually think about the coalgebra itself, rather than the homotopy abelianization of the coalgebra. Indeed, it turns out that if we start from a coalgebra, then there is no need to consider the homotopy abelianization of the coalgebra itself: instead, we may want to define a new coalgebra by replacing all of the objects of the coalgebra by another collection of objects. An easy example of this kind of problem is the classical algebra of groups and sheaves (see~\cite[Section 7.3]{Witten2021}). 

One way to fix this problem is to assume that $A$ is an algebraic structure over $\mathcal{C}$ and define a new coalgebra. The new coalgebra $A' = A^* \colon X \otimes Y \to \mathcal{C}$ will then be an algebraic structure over $\mathcal{C}'$ whose morphisms will be called \textit{projections}. Projections over the object $X$ will turn out to be morphisms over $Y$, and hence we end up with a new algebra over $\mathcal{C}'$. Recall that this construction is known as the Homological Abelianization~\cite{Witten2021}. Here $Y$ will be assumed to be a finite dimensional vector space.

Now, we define a functor $U: \mathcal{C} \to \mathcal{C}'$ by taking the full subcategory of $\mathcal{C}$ consisting of objects that form an object of $\mathcal{C}'$. The composition with $U$ will end up being a map $A' \to A'$ (recall that this is just an equivalence over $\mathcal{C}'$) inducing a functor
\[
U^* \colon \mathcal{C}' \to \mathcal{C}
\]
which will give us a functor 
\[
F: \mathcal{C} \to \mathcal{C}'
\]
from the category of algebras in $\mathcal{C}$ to the category of algebras in $\mathcal{C}'$ (up to homotopy between projections and objects). Observe that if there is a collection of objects $\{V \mid V \in \mathcal{C}\}$ then $\{V' \mid V' \in \mathcal{C}'\}$ will always be equivalent to the same collection of objects in $\mathcal{C}'$ (this can happen in general). This implies that $U$ is essentially surjective on objects and preserves projectives.

\subsubsection{Examples of Coalgebras}

For example, suppose that $A = \mathbf{Set}$ (or another category) and $U = \mathrm{Id}: \mathbf{Set} \to \mathbf{Set}$, that is, $U(X) = X$ for any $X \in \mathbf{Set}$, and $U(A) = A$. Then, the coalgebra of algebras in $A$ is the classical group (if the group laws hold) and its homotopy abelianization will be the unit group. Hence, we have the following coalgebra:
$$A \cong G^{(+)}(\mathrm{Id}: \mathbf{Set} \to \mathbf{Set}).$$

Let $U' := \mathrm{id}: \mathcal{C} \to \mathcal{D}$ be the inclusion functor and recall that $U'(\mathcal{C}) = \mathcal{C}$ and $U'(\mathcal{D}) = \mathcal{D}$. Since $U'$ is essentially surjective, it gives rise to a functor $G' := U'^\ast: \mathcal{C} \to \mathcal{D}$ by precomposing with $U'$. The composition with $G'$ will end up being a morphism $\mathcal{D} \to \mathcal{D}$ that sends $x \in \mathcal{C}$ to $x \in \mathcal{C'}$, and hence we obtain a functor $F := G'^\ast: \mathcal{C} \to \mathcal{D}$. Here $\mathcal{C}$ will be assumed to be a category and $\mathcal{D}$ a category. Since $F$ takes into account injectivity of $U$ (since injectivity of an inclusion is equivalent to injectivity of its adjoint), $F$ can be shown to be a functor $G'' := F^\ast: \mathcal{D} \to \mathcal{C}$. As before, this corresponds to a functor $G': \mathcal{C} \to \mathcal{C}$. Now, since $G' = G''^\ast: \mathcal{C} \to \mathcal{D}$, we get a functor $F := F'^\ast: \mathcal{D} \to \mathcal{C}$. By precomposing with $F$, $F$ ends up sending $c \in \mathcal{C}$ to $c' \in \mathcal{D}$ (because of the fact that $F$ preserves injectives) and hence $\mathcal{C}' = G' \circ F'$.

\subsubsection{Conclusion}

While it is possible to construct a coalgebra over an algebraic structure in a non-abelian sense by introducing the definition of a new algebraic structure over a different category, in this article, we have seen how the coalgebra of algebras can be defined in a general case. To understand the coalgebra of algebras in a certain abelian category, we are interested only in its homotopy abelianization; this is where the coalgebra of algebras becomes critical. We will therefore focus on showing that coalgebras exist.

We now consider the homological abelianization of the coalgebra, namely, $R = R_{ab} \colon X \otimes Y \to \mathcal{C}$ that we mentioned earlier. This is also the object of our interest in this article. We will be able to define a functor $H := H_{ab}: R \to \mathcal{C}$ by making use of the Homological Abelianization~\cite{Witten2021}. For a category $\mathcal{C}$, its homological abelianization is induced by the tensor product, $\otimes$, of $\mathcal{C}$. Thus, $R_{ab} \otimes S$ is the category of functions $X \otimes Y \to Z$ such that for every $x \in X$ and $y \in Y$ there exists an object $z \in Z$ that is both $x$ and $y$ together. For a morphism $h: x \otimes y \to z$ we denote by $F_h: R_{ab} \otimes S \to R_{ab} \otimes T$ the map obtained by composing $h$ with $R_{ab}$. Observe that $F_{F_h}: R_{ab} \otimes S \to R_{ab} \otimes T$ coincides with $F_{h}: X \otimes Y \to Z$ in $\mathcal{C}$.

By defining a functor $H$ as above, we were able to define $H$ as a composite of functors $G$ and $F$ that was shown to be a functor $H = G \circ F$. This completes the proof.

\section{The Problem of Algebraic Categories Over Categories}

This section aims to answer some questions we encountered when working on the question of algebraic categories over categories~\cite{Klein2007}. We will first discuss some of the main examples of problems involving algebraic categories over categories and see if it makes any difference whether one works over a bicategory or a \textit{symmetric monoidal category}. Next, we will give a survey of examples of algebraic categories over categories we found interesting but didn't know how to describe. Finally, we move to the categorical perspective. We will state some profound questions that we hope will prove useful.

\subsection*{Problems with Categories}

Some of the problems discussed here are closely related to problems in categories: specifically:
\begin{itemize}[leftmargin=-0.6cm]
    \item $\mathcal{C}$ is the category of small categories, whereas most algebraic structures on $\mathcal{C}$ are assumed to have $\otimes$ as the tensor operation (for instance, the category of sets).
    \item There is no ``universal" category, namely, no category such that $\otimes$ has a unit (such as $\mathbb{N}^{op}$); however, one can ``define" universal categories via the notions of strict and weak versions: 
    \begin{enumerate}[leftmargin=*]
        \item The notion of \textit{strict} categories is based on categories, and we will often use it to mean that the \textit{strong} version of the tensor product has a unit (when it exists).
        \item The notion of \textit{weak} categories is defined on categories by assigning to an object $X$ a new category $\mathcal{C}_X$ with $\otimes$ as its tensor product. When $X$ is the unit object $X$ of $\mathcal{C}$, this corresponds to the notion of a \textit{strict} functor (with unit $X$ and the tensor product function given by taking the image under the canonical isomorphism).
    \end{enumerate}
\end{itemize}
Note that $\mathcal{C} = \mathcal{S}$ and $\mathcal{C}_{X} = \mathcal{S}_X$, i.e., the category of small functors from $X$ to themselves. This results in the categories of coalgebras over algebras and functors. However, we did not see any references in the literature that included details on the definitions of these algebras and functors, but our arguments and intuition should give the reader some insight.

The above definitions lead to a variety of different results, but we are going to focus on the ``straightforward" result.

\begin{theorem}
    The forgetful functor $U \colon \mathcal{C} \to \mathcal{C}$ induces a functor $U' := U^\ast: \mathcal{C} \to \mathcal{C}'$ over a different category $\mathcal{C}'$.
\end{theorem}

\begin{proof}
    Suppose $\mathcal{C} = \mathcal{S}$. Then we have
    $$U^\ast \colon \mathcal{S} \to \mathcal{S} \cong \mathrm{Id}: \mathcal{S} \to \mathcal{S},$$
    which is the identity on categories. Since $U$ is the forgetful functor, we have
    $$\mathcal{S} \cong U^\ast^{-1}(\mathcal{S}),$$
    which is a strong monoidal category. We show that $\mathcal{S} \cong U^\ast^{-1}(\mathcal{S}') \cong \mathcal{C}$. The first assertion is clear. The second follows from the fact that, whenever $X$ is a small category, the object $X^*$ is in the homotopy category $\mathcal{C}_{X}$. By the same argument, we get that the functor $X \otimes -: \mathcal{C} \to \mathcal{C}$ extends to a functor $X^*\otimes -: \mathcal{C} \to \mathcal{C}'. Moreover, we get the following square in $\mathcal{S}$:
    $$
        \begin{tikzcd}
            X \otimes X^* \arrow[r, shift left=0.2cm,"{X \otimes -}"'] \arrow[d, shift left=0.2cm,"\otimes"] \arrow[rd, phantom, "\scalebox{0.9}{$\scriptstyle {\hookrightarrow}$}"] & X \arrow[l, shift left=0.2cm, "{\otimes}"] \arrow[ld, phantom, "\scalebox{0.9}{$\scriptstyle {\longrightarrow}$}"] \\
            X^* \arrow[shift left=0.2cm, dashed, "X^{\otimes}"] \arrow[u, dashrightarrow, shift left=0.2cm, "{\otimes}"] 
        \end{tikzcd}
    $$
    By adjunction, this is equivalent to the fact that $X \otimes -$ is a functor between strict and weak (strict) categories and that $X$ is equipped with a canonical unit $X^{\otimes} = X$ (thus making $X^{\otimes}$ the same as $X$). Therefore $X \otimes -$ is induced by an embedding of $\mathcal{C}$ into $\mathcal{C}_X$. In particular, $X \otimes -$ is fully faithful.

    Now suppose that $\mathcal{C} = \mathcal{D}$. Let $\mathcal{C}' = \mathcal{E} \boxtimes_{\mathcal{C}} \mathcal{D}$ be the category of $2$-categories with the tensor product induced by the symmetric monoidal structure on $\mathcal{C}$. Since $\otimes$ is the tensor product in $2$-categories, this defines a functor $U' := U^\ast: \mathcal{D} \to \mathcal{E} \boxtimes_{\mathcal{C}} \mathcal{D}$; the inverse functor simply exhibits the functor $U$ on objects. The second assertion follows because $U'$ is essentially surjective on functors and preserves coproducts (when the categories involved admit coproducts). For a functor $F := (F_p, F_q) \colon X \to Y$ we obtain a natural isomorphism 
    \[ F^\ast_{(p,q)} \colon U'_{(p,q)}(F)(X) \to U'_{(p,q)}(F)(Y). \]
    One obtains that $F^\ast_{(p,q)}$ commutes with $\otimes$, as desired.
\end{proof}

The following statement is well-known in the world of categories, even though it appears in a context of homotopy categories (which the authors have removed the reference from).

\begin{corollary}
    Let $A$ be an algebraic structure on $\mathcal{C}$ and $\mathcal{C}'$ be another category. The category of $A$-modules over $\mathcal{C}'$ is equivalent to the category of modules over $\mathcal{C}$, and it is also equivalent to the category of (strong) functors from $A$-modules to $A$-modules over $\mathcal{C}$.
\end{corollary}

\begin{remark}
    Notice that it does not make much sense to define a functor from a module to another module. The point here is to show that if $A$ is a module, then there is a functor $M: A \to \mathcal{C}$ that defines a \textit{$A$-module} over $\mathcal{C}$. Indeed, it turns out that the adjunction $A \dashv M$ is effectively transparent and therefore induces a $A$-module functor $M: A \to \mathcal{C}$. The reason for this was due to the fact that, in our setting, the category $\mathcal{C}$ already includes many other information and $M$ cannot be defined as a functor from a module to itself without rewriting it explicitly (except possibly for a specific choice of a module). This is generally not the case when working over bicategories, where $M$ needs to be defined as a functor over a bicategory as opposed to a functor over a symmetric monoidal category. In the next paper, we will take a step forward and move towards understanding the homotopy categories of functors instead of the algebras. 
\end{remark}

In this paper, we introduce the notion of a \textit{coalgebra} of a coalgebraic structure on a category $\mathcal{C}$. We also review some of the basic properties of a coalgebra over a certain category (which we expect to be particularly useful later on).


\section{Algebras vs Coalgebras}

Let us begin by explaining briefly why coalgebras are important in mathematics. They play a central role in the modern mathematical theory of algebras. It seems very natural to think about coalgebras as a generalisation of coherent algebras. First of all, while coherent algebras are still in use today, they can also be reformulated in terms of coalgebras. Recall that two algebras $A$ and $B$ are equivalent if $A$ and $B$ are compatible with respect to the tensor product, so that $A \otimes B$ is also compatible. Let us define the category of compatible algebras in this sense.

\begin{definition}
    Let $A$ and $B$ be two coherent algebras. A \textit{compatible pair} is an equivalence relation satisfying the conditions given below: 
    \begin{enumerate}[leftmargin=*]
        \item[(a)] Any equivalence class of $A$ contains the same set of objects as $B$ and $A$ is isomorphic to $B$.
        \item[(b)] Given an element $y \in B$, if $y \in A$ then so is $y \otimes B$.
    \end{enumerate}
    
    An equivalence class $C$ of $A$ is said to be \textit{compatible with respect to the tensor product} if:
    \begin{enumerate}[leftmargin=*]
        \item[(i)] $A$ is isomorphic to $B$.
        \item[(ii)] Every pair of objects in $B$ has an isomorphism to the identity object $1$.
    \end{enumerate}
\end{definition}

We will now introduce the coalgebra structure and show it is compatible with respect to the tensor product.

\begin{lemma}[Completeness]
    Let $A$ be a coherent algebra. The induced homotopy category of compatible pairs in $A$ is isomorphic to the homotopy category of compatible pairs in $\mathrm{Hom}(A,-)$ for any left adjoint $L$ of $A$. 
\end{lemma}

\begin{proof}
    Suppose $A$ is compatible. We define the homotopy category $\mathrm{Hom}(A,\mathrm{Id}):=\mathrm{End}(A)$ as the full subcategory consisting of $A$-modules and isomorphic $A$-modules. Let $X \in \mathrm{Hom}(A,\mathrm{Id})$ be a module. Since $A$ is compatible with respect to the tensor product, we have $X \otimes B = L(X \otimes B)$. Then the induced functor $L:A \to \mathrm{Hom}(A,\mathrm{Id})$ is an equivalence and therefore so is the functor $L$. From Proposition~\ref{functor equiv}, we conclude that $L$ is an equivalence in $\mathrm{Hom}(A,\mathrm{Id})$. Therefore, the inclusion $\mathrm{Hom}(A,\mathrm{Id})\hookrightarrow \mathrm{End}(A)$ sends a $A$-module $M$ to the endofunctor $L(M)$ of $A$ (whose domain is $A$). Since $L$ is fully faithful, this yields a functor $L:A \to \mathrm{End}(A)$. 
    
    Note that $L(M)$ is not necessarily injective. Indeed, by assumption, there are an equivalence classes $I_{x} \in \mathrm{End}(A) \subseteq \mathrm{Hom}(A,\mathrm{Id})$ such that $L(X \otimes I_{x}) = X$ for all $X \in \mathrm{End}(A)$. Then there is an injection $I_X \in \mathrm{Hom}(A,\mathrm{Id})$ such that $L(I_{x} \otimes X) = I_{x}$ for all $x \in A$. This induces a natural isomorphism 
    \[ L^{-1}(X) = X \otimes I_X.\]
    Therefore $L(M) = M \otimes I_M$ where $M \otimes I_M$ is the same as $M$, but in $L^{-1}$. Furthermore, $L$ is a monad and hence preserves the counit and multiplication.
    
    Since we have proven the isomorphism above, there are an isomorphism of categories $\mathrm{Hom}(A,\mathrm{Id}) \simeq \mathrm{End}(A)$ between the categories of compatible pairs and the category of modules over $\mathcal{C}$.
\end{proof}

\begin{definition}[Modules over coherent algebras]\label{def module coherence}
    Let $A$ be a coherent algebra. A \textit{coherent $A$-module} is an object $M$ of the induced homotopy category of compatible pairs in $A$. Its image in the category of modules over $\mathcal{C}$ is called its \textit{homology} or its \textit{derived homology}.
\end{definition}

\begin{example}\label{ex modules over coherent algebras}
    Let $A = \mathbf{Set}$ be the category of sets and let $\mathbf{Mod}(A)$ denote the category of coherent $A$-modules. Recall that $A$ is the category of finite sets and any coherent $A$-module is an object in $\mathbf{Mod}(A)$. This has an exact embedding $A \hookrightarrow \mathbf{Mod}(A)$ and hence is a monoidal category, in which case $A$ is a coherent $A$-module. In particular, $A$ is a coherent $A$-module.
\end{example}

We can now consider the ``strong" version of the tensor product: if $B$ is a coherent $A$-module and $M$ is a coherent $B$-module, then $M$ is called a \textit{coherent $A \otimes B$-module}, and the tensor product $A \otimes B$ is called the \textit{coherent $A \otimes B$-coproduct}. The coherent $A \otimes B$-coproduct is defined as the image of the equivalence class of the derived homology $H^1(A \otimes B, \mathrm{Hom}(A,\mathrm{Id}))$. That is, in fact the coherent $A \otimes B$-coproduct for any $y \in B$ is given by the derived homology of the isomorphism class $A \otimes B$ by $y \in A \otimes B$.

The above definition and remark provide an explicit description of the homology of a coherent $A$-module $M$. The homology of the coherent $A$-module $M$ over the coherent algebra $A$ will be called its \textit{derived homology}.

It is clear that the tensor product of coherent $A$-modules is coherent $A \otimes B$-modules. We will refer to the category of coherent $A$-modules over $B$ as the \textit{category of modules over $B$}, and we call this the \textit{category of $A$-modules over $B$}. 

Let us revisit our previous definition of a coalgebra over $A$. As explained above, if $B$ is a coherent $A$-module then so is the coherent $A \otimes B$-module, and we call the coherent $A \otimes B$-coproduct the coherent $A \otimes B$-module's derived homology, or the \textit{coherent $A \otimes B$-coproduct}.

Let us fix $B$ to be a coherent $A$-module. The homology of a coherent $A$-module $M$ will be referred to as its \textit{$A$-homology} (after an acronym for its derived homology). An $A$-homology $A' \to A$ is a family of functions $h_k: A_k \to A$ such that $h_k$ is a coherent $A$-module homomorphism over the coherent algebra $A$. The category of $A$-homology will be called the \textit{category of $A$-modules over $B$}. 

Let us now explain how we obtain the homology of coherent modules over $B$ from coherent ones over $A$. There are several ways to solve this problem.

\begin{method}
	The solution to the problem of obtaining the $A$-homology of a coherent $A$-module is the following: given a coherent $A$-module $M$, we consider the $A$-homology of $M$ over $A$ and compute the derived homology $H^k(M,A)$ of $M$. This provides the $A$-homology of a coherent $A$-module $M$ as the quotient by a coherent $A$-module homomorphism $A'\to A$ that takes each coherent $A$-module to an $A$-module over $A$. To do so, we will follow a certain principle: first observe that the derived homology of $M$ over $A$ agrees with the derived homology of a coherent $A$-module $N$ over $A$. So we will replace $N$ by $H^k(M,A)$ where $k$ is any integer. The coherent $A$-module homomorphism associated to the $A$-homology $A' \to A$ will then be replaced by the $A$-homology $A' \to A$ of the coherent $A$-module $N$. The quotient then replaces $N$ by $H^k(M,A')$. In particular, the $A$-homology $A' \to A$ of a coherent $A$-module $N$ will be obtained from the $A'-H^k(N,A)$ and $A$-homology $A' \to A$ of the coherent $A$-module $M$ using the following formula: 
	$$H^k(M,A') = \frac{H^k(M,A')}{\textnormal{coker}( h^k \circ I_{A'})}.$$
	Let us also recall the following lemma:
	
	\begin{lemma}\label{lem coherent homology of module}
		Consider a coherent $A$-module $M$ over $A$ and let $A$ be a coherent algebra. Then the coherent $A$-module homology of $M$ is equal to the $A$-homology of $H^k(M,A')$ for any $k \in \mathbb{N}$.
	\end{lemma}
	
	\begin{proof}
		First, observe that $\textnormal{coker}(-
\end{document}
):=\textnormal{ker}(-) \subseteq \textnormal{im}(-)$ is an equivalence in the category of coherent $A$-modules over $A$. Therefore, by lemma~\ref{lem coherent homology of module}, there exists a unique $A$-module homomorphism $\mathrm{Im}(-) \to \textnormal{ker}(-)$ such that $\mathrm{Im}(-)$ is an $A$-module homomorphism over the coherent algebra $A$. So the coherent $A$-module $H^k(M,A')$ is the quotient by this $A$-module homomorphism by the $A$-homology $A' \to A$. 
		
		Next, observe that $A \otimes B$ is a coherent coproduct of $A$-modules over $B$. As an endofunctor, $B$ is the restriction of $A$ to the subset $\mathrm{ker}(-)$ of coherent modules over $B$, which is coherent $A$-modules over $A$. So $B$ is compatible with respect to the tensor product and thus has the homology of the coherent $A$-module $