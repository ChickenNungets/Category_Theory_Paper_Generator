
\documentclass[a4paper,reqno,oneside]{article}
\pdfoutput=1
\include{mathcommands.extratex}
\begin{document}
\title{Functorialization Of Higher Topos And Their Geometry}
\author{Max Vazquez}
\maketitle


We will start by constructing a functorialized version of a topos via
higher categorical structures, and then show how the induced functor can be
used to construct higher categories from other categories. In addition, we
show that this construction is compatible with the adjunctions between categories.

\section{The Definition Of A Topos}
Let us begin by reformulating the notion of \emph{topos} using a classical
notation. 

Recall that an object $X$ in a category $\CAT{C}$ consists of all 
\emph{objects} in the category and all \emph{morphisms} between them; a morphism 
between objects $(A,f)$ in $\CAT{C}$ and $(B,g)$ in $\CAT{D}$ consists of a 
function $f: A \rightarrow B$ in $\CAT{C}$ and a function $g: f^* B \rightarrow A$ in
$\CAT{D}$, where the composition is given by the left unitor \eqref{eqn.uleftunitor}. 

We introduce the following notation for $\CAT{C}$. 

\begin{definition}[{\cite[Def 11.6]{heister2017}}]
An object of $\CAT{C}$ is called \emph{left adjoint} if there exists 
a function $f: C \rightarrow D$ such that the diagram below commutes: 

\[
    \xymatrix@R-3cm@C-3cm{
        & {\CAT{C}^{\op}} \\
        {X} && {D}
        \ar@{->>}[ru]^{f} 
        \ar@{<->}[ru]_{} 
    }
\] 
\end{definition}

Note that we have an easier (but still not natural) notion of an object
in $\CAT{C}$ when we consider objects that we may regard as maps. This is the
same definition used by the Eilenberg--Moore category. 

By the dual to the previous point, an object of $\CAT{C}$ is called
\emph{right adjoint} if it admits a map $r: X \rightarrow D$, which is also
called a \emph{unit}, for every morphism $f: C \rightarrow D$.  The same
argument is made in the opposite direction, where right adjoints are given
by objects with a right unit. Here is another example:

\begin{definition}[{\cite[Def 11.5]{heister2017}}]
An object of $\CAT{C}$ is called \emph{coadjoint} if it admits a map $r: D \rightarrow C$,
which is called a \emph{counit}, for every morphism $f: D \rightarrow C$.  The same
argument is made in the opposite direction, where coadjoints are given
by objects with a counit.
\end{definition}

For each right adjoint $(D,r)$ and coadjoint $(C,c)$, one defines a \emph{homotopy 
comonoid} $\Hom(D,C)$ consisting of all the maps which satisfy
\emph{isomorphisms}. 

\begin{definition}[{\cite[Sec 18.1]{heister2017}}]\label{defn.htpcats}
For any right adjoint $(D,r)$ and coadjoint $(C,c)$, define the \emph{homotopy 
category} $\CAT{D}^{+}(C) \subset \CAT{D}$ as follows:
\[
(\Hom(D,C))_0 = \{x : x\text{-}\in D\} \cup (\Hom(D,C))_1 = 
\{y : y \text{-}\in D\}^{\times 2}.
\]
A map $g : D^{+} \rightarrow \Hom(D,C)$ is an \emph{equivalence} if $g(x,\dots,x)$
and $g(y,\dots,y)$ for all $x,\dots,x$ and $y$ in $\Hom(D,C)_0$.
\end{definition}

\begin{remark}
If $(D,r)$ and $(C,c)$ are two homotopy categories, then their \emph{multiplication} 
is given by $g^{\bullet}: C \rightarrow \Hom(D,C)^{\times 2}$, and its \emph{addition} 
is given by $g_{\bullet}: D \rightarrow \Hom(D,C)^{\times 2}$.
\end{remark}

For a right adjoint $(D,r)$, let $c_0$ be the component corresponding to $0$ and
$c_1$ the component corresponding to the first summand $x + 0 = c_0 + r(x)$,
and so on for $1$. By definition, $c_{|1} \in \Hom(D,C)^{\times 2}$ and
$c_1,c_{|2} \in \Hom(D,C)^{\times 2}$. Then for any $n \geq 0$, define 
$\Hom(D,C)^{\times n} = \sum_{i=0}^n c_{|i} \in \Cat{D}$, so that
$$
\Hom(D,C)^{\times n} = \{0:\cdots 0,\ldots, c_1\cbr 0, c_{|2} 
\cbr 0,\ldots c_{|i-1}\cbr i,\ldots, c_{|n-1}\cbr i, c_0\cbr i \}
$$
The component $c_n$ corresponds to $x + r(x)^n$.

To get a homotopy category, it suffices to take these components as homotopy 
colimits, up to coarsest representable level. For instance, if $\CAT{D}$
has two components $d_0$ and $d_1$ whose representation is a double group, 
then we can take $c_0:=d_0,c_1:=d_1$ and obtain a homotopy category 
of the form
\[\begin{tikzcd}
	& {\CAT{D}^{+}} \\
	d_0 && d_1
	\ar[rr] \ar[dd] 
	&& {d_0^{+}, d_1^{+}}.
	\arrow["d_0", shift left=2, from=1-3, to=3-3] 
	\arrow["d_1"', shift left=2, from=1-3, to=3-3]
\end{tikzcd}\]
The structure of a group homotopy category is closely related to that of a
double group homotopy category. Let $\CAT{G}_R := \CAT{D}^{+}(R)$ be the homotopy 
category defined above. If $R$ is the ideal generated by $k_0, k_1, k_2, \dots$, then 
we define the group homotopy category $\CAT{G}_R := \CAT{D}^{+}(R)$ to be the
group homotopy category with a single component $c_1$ constructed from the 
component $c_0$ and the components $k_0$ and $k_1$ respectively, namely 
$$\begin{tikzcd}
	& {\CAT{D}^{+}} \\
	c_0 && k_1.
	\ar[rr] \ar[dd] 
	&& k_0^{+}.
	\arrow["c_0", shift left=2, from=1-3, to=3-3] 
	\arrow["k_1"', shift left=2, from=1-3, to=3-3]
\end{tikzcd}$$ 
The structure of a double group homotopy category can be understood as that of 
a double group homotopy category with the same components except for the coadjoint $c_0$ 
whose value lies in the coadjoint group $R^{n-1}$, namely
$$\begin{tikzcd}
	& {\CAT{D}^{+}} \\
	c_0 && {c_1^{+}}
	\\
	c_1 && {}
	\ar[rr] \ar[dd] 
	&& {}
	\arrow["c_1", shift left=2, from=1-3, to=3-3] 
	\arrow["{c_0^{+}}"', shift left=2, from=1-3, to=3-3]
\end{tikzcd}$$ 
The coadjoint groups $R^{n-1}$ are generated by $k_0, k_1, k_2, \dots$.  We now 
define the structure of the dual group homotopy category $\CAT{G}_R^{-1} :=
\CAT{D}^{+}(R^{-1})$. Its component $c_1$ is obtained by composing the component 
$c_0$ with the inverse image of $k_0$ and $k_1$ in $R^{-1}$. The components $k_0$
and $k_1$ are thus inverse images of $k_0$ and $k_1$ respectively, since the
dual group homotopy category is equivalent to the homotopy group homotopy category
with components $c_1$ being those images of $c_0$. The component $c_1$ is again 
defined by composing the component $c_0$ with the inverse image of $k_0$ and $k_1$ in $R^{-1}$.
In this way, we obtain the dual structure on the homotopy category by composing 
the structures of the homotopy categories, namely by composing the homotopy structure 
from the category $\CAT{D}^{+}(R)$ to the one of the dual group homotopy category 
$\CAT{G}_R^{-1}$, as well as by composing the coadjoint structure of the dual 
group homotopy category $\CAT{G}_R^{-1}$ to the one of the homotopy category.

Finally, note that we require a special condition. Since $R$ is an ideal, we may 
always associate an invertible 3-cocycle $u$ into the homotopy category $\CAT{G}_R$. 
This defines an invertible isomorphism $R^{n-1} \cong R$ such that
$u(i,j) = (i^{*} = j^{*}), i \in R^{n-1}, j \in R^{n-1}$ for $i < j$, or, equivalently,
$u(i,j) = i - j$, for some $i < j$. It is easy to see that $u$ must be an 
invertible 3-cocycle because then $R^{n-1} \cong R$ for all $n > 0$. 

Our next goal is to define our notion of topos based on the new notion of
"homotopy category".

\begin{definition}[{\cite[Def 16.15]{heister2017}}]\label{defn.htp}
Let $R$ be an invertible $n$-cocycle in $\mathcal{C}$. Define the \emph{topos} of 
an ideal $R$ in $\mathcal{C}$ as the \emph{monoid of discrete functors} 
defined as follows:
\begin{enumerate}[(i)]
\item A left adjoint $\tau_R : R \rightarrow R$ is said to be a \emph{discrete functor} if it has the property that $R \tau_R \tau_R = R$ for all $n > 0$.
\item A coadjoint $\eta_R : R \rightarrow R$ is said to be a \emph{discrete equivalence} if it satisfies the property that $R \eta_R \eta_R = R$ for all $n > 0$.
\end{enumerate}
An object $A$ in $\mathcal{C}$ is said to be \emph{co-left adjoint} if it 
satisfies the property that $A \circ \tau_R = A$ and $A \circ \eta_R = A$.
An object $B$ in $\mathcal{C}$ is said to be \emph{co-right adjoint} if it
satisfies the property that $B \circ \tau_R = B$ and $B \circ \eta_R = B$.
\end{definition}

Note that $\mathcal{C}$ is necessarily closed under left adjoints and coadjoints, 
as well as monoidal and coherence functors. Furthermore, todays mainstream textbooks 
are only concerned with the coadjoints and monoidal functors. Therefore, we can simply 
consider the notions of discrete functors, discrete equivalences, and co-left adjoints 
and co-right adjoints. 

By definition, the topos $\mathcal{T} := \mathcal{T}^L = \mathcal{T}^C$ consists of the discrete 
functors together with the discrete equivalences and co-left adjoints together.
We write $\CAT{T} := \CAT{T}^L \cup \CAT{T}^C$ for the full subcategories of $\CAT{T}^L$ 
consisting of the coadjoints and monoidal functors, respectively, and 
respectively. 


As shown in Remark \ref{rem.functoftopos}, the coadjoints are unique up to natural isomorphism, 
and conversely, the monoidal functors are unique up to natural isomorphism. 

We also want to describe the functors taking the objects and arrows of $\mathcal{T}$ to a variety of other structures, including: 

\begin{itemize}
\item The set of arrow $\tau$ valued functors $\tau_R$ with left adjoints, 
\item The set of monoidal equivalences $\eta_R$ with right adjoints, 
\item The set of coadjoints, 
\item The set of co-left adjoints.
\end{itemize}
We call such objects the \emph{adjunctions}. The \emph{functorialization} we will provide is precisely 
one of the adjunctions.


\subsection{Higher Topos}
We begin by introducing the concepts of \emph{higher categories}, 
\emph{higher functors}, and \emph{higher adjunctions}. These will be used later in the construction.

\begin{definition}
Let $\CAT{E}$ be a category, with two classes of objects, $E_0$ and $E_1$. An \emph{$E_0$-enriched category} is a category enriched over $\CAT{E}$
such that every object in $\CAT{E}$ is an $E_0$-enriched object. 
\end{definition}

In this case, the functor $\CAT{F}$ extends to an $E_0$-enriched functor between enriched categories as follows:
$$\CAT{F}(\one,B) := E_{0} \leftarrow \CAT{F} (B,0).$$
By defining $\tau_R := \CAT{F}(\one,B)$ as $\cat{I}$, the functor $\tau_R$ is essentially a special kind of enrichment that comes with a fixed adjunction $\tau_R \dashv \eta_R$. 

If $\CAT{E}$ is an $E_1$-enriched category, then we consider $E_0$-enriched categories as such enriched 
categories with the additional property that $\tau_R$ is an $E_1$-enriched isomorphism as in 
\begin{equation}\label{eqn.enrichedisomorphism1}
\tau_R \circ \tau_R = \tau_R \circ \eta_R = 
\tau_R \circ \eta_R \circ \tau_R = \tau_R \circ \eta_R \circ \eta_R = \eta_R \circ \eta_R \circ \tau_R = \tau_R \circ \eta_R = \tau_R.
\end{equation}

If we were looking at a category $\CAT{E}$ that has both $E_0$ and $E_1$-enriched 
objects, then a generalised example of $E_0$-enriched categories is the category $\CAT{Vect}$
of vector spaces, which is the category of algebras over an algebraically closed field $A$ equipped with a functor
$\sigma_A: A \rightarrow A^*$ that sends each vector space $V$ to the $A$-vector space associated to the unit $a \in A$. If this $a \in A$, we say that $V$ is \emph{continuous}. 

\begin{definition}
Let $A$ be an algebra over an algebraically closed field. The \emph{$E_1$-enriched category} $\CAT{E}_1$ is the $E_1$-enriched category of vector spaces over $A$. 
\end{definition}

Therefore, if we replace $\CAT{Vect}$ with $\CAT{E}_1$, then we will now refer to
this category as the \emph{vector space category} associated to $A$. Let $\twovec$ denote the category 
generated by $2$ vector spaces. Similarly, we can write $\cat{S}$ for the $E_1$-enriched category generated by an arbitrary sequence of finite sets, where the arrows correspond to the maps in $\twovec$. 

The enriched version of the notion of $E_1$-enriched category was introduced by
Herschend in \cite{herschend2020higher}, but it was changed slightly for 
our purposes here. The original definition uses $A$ as the base, whereas we will 
now treat $\CAT{E}_1$ and $\cat{S}$ as having the same underlying category $A$.

\begin{definition}[{\cite[Def 16.9]{herschend2020higher}}]
Let $\CAT{E}$ be an $E_0$-enriched category. An $E_1$-enriched category is an $E_1$-enriched category $\CAT{E}$ where the two classes of objects consist of finite sets rather than sets of spaces.
\end{definition}

\begin{definition}[{\cite[Def 16.10]{herschend2020higher}}]\label{defn.htp}
Let $\CAT{C}$ be an $E_1$-enriched category. We call an object $X$ of $\CAT{C}$ 
\emph{locally $\tau_R$-adjoint} if it admits a local $E_1$-enriched isomorphism $\tau_R \colon 0 \rightarrow X$
satisfying the following conditions:
\begin{enumerate}
\item If $\tau_R$ is an $E_1$-enriched isomorphism, then the pair $(\one,-)$ induces a local equivalence $\one \stackrel{\tau_R}{\rightarrow} X$.
\item If $\tau_R$ is a local equivalence, then the pair $(\one,A)$ induces a map $X \rightarrow \cat{S}$ in $\CAT{C}$ sending each space $V$ to its finite set of permutations of $A$.
\item If $\tau_R$ is a local map, then the pair $(\one,A)$ induces a map $X \rightarrow \cat{S}$ in $\CAT{C}$ making the following diagram commute:
$$\begin{tikzcd}
    0 \arrow[swap,"0"] 
     & A \arrow[swap,"A"] 
     \\
    X 
     & \cat{S}
     \arrow["t"]
\end{tikzcd}$$ 
where $t$ is the $E_1$-enriched functor that takes the space $A$ to the space of permutations of $A$.
\end{enumerate}

An object $Y$ of $\CAT{C}$ is \emph{locally $\eta_R$-adjoint} if it admits a local $E_1$-enriched 
isomorphism $\eta_R \colon Y \rightarrow 0$ satisfying the following conditions:
\begin{enumerate}
\item If $\eta_R$ is an $E_1$-enriched isomorphism, then the pair $(-,A)$ induces a local equivalence $X \stackrel{\eta_R}{\leftarrow} 0$ (note the flip).
\item If $\eta_R$ is a local equivalence, then the pair $(-,A)$ induces a map $\cat{S} \rightarrow A$ in $\CAT{C}$ sending each permutation of $A$ to the space $V$.
\item If $\eta_R$ is a local map, then the pair $(-,A)$ induces a map $\cat{S} \rightarrow A$ in $\CAT{C}$ making the following diagram commute:
$$\begin{tikzcd}
    0 
     & \cat{S} \arrow[swap,"\cat{S}"] 
     \\
    A 
     & \cat{S}
     \arrow["t"]
\end{tikzcd}$$ 
where $t$ is the $E_1$-enriched functor that takes the space $A$ to the space of permutations of $A$.
\end{enumerate}
\end{definition}

\begin{definition}[{\cite[Def 16.11]{herschend2020higher}}]\label{defn.htp}
Let $\CAT{E}$ be an $E_1$-enriched category. An $E_1$-enriched functor $\cat{F} : \CAT{C} \rightarrow \CAT{D}$ consists of three parts:
\begin{enumerate}
\item An $E_0$-enriched functor $\cat{F}_0$ such that $\cat{F}_0(B, 0) \cong E_{0}$.
\item An $E_1$-enriched functor $\cat{F}_1$ such that $\cat{F}_1(\one,B) \cong \cat{F}_0(B, 0)$.
\item An $E_0$-enriched equivalence $\cat{F}_{0,1} : \cat{F}_0(B, 0) \rightarrow \cat{F}_1(B, 0)$, which is the action of $\cat{F}_0$.
\end{enumerate}
\end{definition}

These four parts form a symmetric monoidal $\infty$-category $\CAT{D}^E$ such that the equivalences between the parts are $E_0$-enriched. The enriched version of this symmetric monoidal $\infty$-category is referred to as the \emph{symmetric $\infty$-category associated to $\CAT{E}$. 

\begin{definition}[{\cite[Def 16.12]{herschend2020higher}}]\label{defn.htp}
A category $\CAT{E}$ is said to be \emph{locally locally $\tau_R$-adjoint} (resp.\ locally locally $\eta_R$-adjoint) 
if it is locally $\tau_R$-adjoint and locally $\eta_R$-adjoint, i.e., it is both a local $E_0$-enriched category and a locally $\tau_R$-adjoint object of $\CAT{D}^E$ for some symmetric $\infty$-category $\CAT{D}^E$, where $\tau_R$ and $\eta_R$ are local $E_1$-enriched functors. 
\end{definition}

In view of the definition above, we can write this statement much more generally in terms of the monoidal categories $\cat{D}^E$ 
and $\cat{D}^E_\twovec$. This means we need only to consider the symmetric $\infty$-category 
$$\cat{S} = \cat{D}^E_\twovec 
\qquad\text{and}\qquad
\tau_R^\twovect = \tau_R \circ \eta_R = 0 \in \cat{S}
$$
of adjoints and isomorphisms between the objects in the symmetric monoidal $\infty$-category $\cat{S}$.
The enriched version of this symmetric monoidal $\infty$-category is referred to as the \emph{symmetric $\infty$-category associated to $\CAT{E}_1$}. 



\subsection{The Notation Of $E_1$-Categories}
Now, we turn to the main topic of interest.

\begin{definition}
Let $\CAT{D}$ be a symmetric $\infty$-category. We define the \emph{homotopy category} $\CAT{D}^H$ of $\CAT{D}$ as follows:
\begin{enumerate}
\item An object $X$ in $\CAT{D}$ is said to be \emph{right adjoint} if it has a right adjoint $\theta: X \rightarrow X$.
\item An object $Y$ in $\CAT{D}$ is said to be \emph{left adjoint} if it has a left adjoint $\phi: Y \rightarrow Y$.
\item An object $Z$ in $\CAT{D}$ is said to be \emph{co-left adjoint} if it has a co-left adjoint $\kappa: Z \rightarrow Z$.
\item An object $W$ in $\CAT{D}$ is said to be \emph{co-right adjoint} if it has a co-right adjoint $\lambda: W \rightarrow W$.
\item If $X$ and $Y$ are right adjoints, we call $X$ and $Y$ \emph{adjointed} if there exist left adjoints $\theta$ and $\phi$ such that $X \theta = Y \phi$.
\item If $X$ and $Y$ are left adjoints, we call $X$ and $Y$ \emph{co-adjointed} if there exist co-left and co-right adjoints $\kappa$ and $\lambda$ such that $X \kappa = Y \lambda$.
\item If $X$ is left adjointed and $Y$ is co-left adjointed, we call $X$ and $Y$ \emph{adjoint coincide} if there exist right adjoints $\theta$ and $\phi$ such that $X \theta = Y \phi$.
\item If $X$ is right adjointed and $Y$ is co-right adjointed, we call $X$ and $Y$ \emph{co-adjoint coincide} if there exist left adjoints $\theta$ and $\phi$ such that $X \theta = Y \phi$.
\item Let $\alpha: X \rightarrow Y$ and $\beta: Y \rightarrow Z$ be two morphisms in $\CAT{D}$. If $X$ is left adjointed and $\beta$ is co-left adjointed, we call $\alpha$ and $\beta$ \emph{compatible} if there exist compatible morphisms $\alpha' : X \rightarrow Z$ and $\beta'$ such that $\alpha' = \beta'$. 
\item Let $\alpha: X \rightarrow Y$ and $\beta: Y \rightarrow Z$ be two morphisms in $\CAT{D}$. If $X$ is right adjointed and $\alpha$ is co-right adjointed, we call $\alpha$ and $\beta$ \emph{compatible} if there exist compatible morphisms $\alpha' : Y \rightarrow X$ and $\beta'$ such that $\alpha' = \beta'$. 
\item Let $\cat{K} := \bigvee_{r: X \rightarrow Y} r$ and $\cat{K'} := \bigvee_{l: Y \rightarrow Z} l$ be two morphisms in $\CAT{D}$. If $\cat{K}$ is compatible with $\cat{K'}$, we call $\cat{K}$ and $\cat{K'}$ \emph{compatible} if there exist compatible morphisms $\cat{K}': X \rightarrow Z$ and $\cat{K}'': Y \rightarrow X$ such that $\cat{K} = \cat{K}'$. 
\end{enumerate}
\end{definition}

We then define the $E_1$-enriched category associated to a symmetric $\infty$-category $\CAT{D}$ as the homotopy category of $\CAT{D}^H$. Similarly, we write $\CAT{D}^{E_1}$ for the category that has all the objects and morphisms of $\CAT{D}^H$ together with the adjunctions. This allows us to define the $E_1$-enriched category $\CAT{E}_1$ associated to $\CAT{D}$, and moreover we can define two $E_1$-enriched categories in terms of $\CAT{D}$ whenever they admit a right adjoint and a co-right adjoint, respectively.

Thus, the symmetric $\infty$-categories are enriched over a symmetric $\infty$-category $\CAT{D}$ together with the coadjoint and co-adjoint functors which preserve these coadjoint and co-adjoint maps.

\begin{example}
Suppose $\CAT{D}$ is the symmetric monoidal category described in Example \ref{exm.E1_H}, and assume that $\one$ and $\two$ are the two identity objects. 

Then the $E_1$-enriched categories associated to $\one$ and $\two$ are given by the following symmetric monoidal $\infty$-categories: 
\begin{itemize}
\item $\CAT{D}^H_0$ is the symmetric monoidal category with the objects the elements of $\twovec$, the identities $e_0$ and $e_1$, and the morphisms 
$$\begin{tikzcd}
\one & \one \arrow[swap,"e_1"] \arrow[hook,"e_1",swap,shift left=-1pt] 
	\arrow[hook,"e_1",shift left=-1pt] & & e_0 	\arrow[hook,"e_0"]
\end{tikzcd}$$
between the two elements of $\twovec$.

\item $\CAT{D}^H_1$ is the symmetric monoidal category with the objects the elements of $\twovec$, the co-identities $0$, $1$ and $c$, and the morphisms 
$$\begin{tikzcd}
e_0 & 1 & 0 \arrow[swap,"c"] \arrow[hook,"e_0",swap,shift left=-1pt] 
	\arrow[hook,"e_0",shift left=-1pt] & & e_1 \arrow[hook,"e_1"]
\end{tikzcd}$$
between the two elements of $\twovec$.

\item $\CAT{D}^H = \CAT{D}^H_0 \cup \CAT{D}^H_1$ is the symmetric monoidal category with the objects the union of all objects of $\CAT{D}^H_0$ and $\CAT{D}^H_1$ together with the functors $\tau: \twovec \rightarrow \twovec^2$ and $\eta: \twovec^2 \rightarrow \twovec$ respectively which preserves the inclusion of $c$ in $\twovec$.

\end{itemize}

It is straightforward to check that the category $\CAT{D}^{E_1}$ is a symmetric monoidal $\infty$-category. Note that $\twovec$ is indeed a symmetric monoidal category by \cite[Rem 10.3]{haugseng2006higher}. The category $\cat{S}$ is the symmetric monoidal category with the same objects and morphisms as $\cat{D}$ but for an endomorphism $X \rightarrow Y$ in $\CAT{D}$ with co-adjoints $\kappa$ and $\lambda$, such that $\cat{K}' = (\kappa \otimes \lambda)^\twovect$ and $\cat{K} = \cat{K}': X \rightarrow Z$ if and only if $\kappa$ and $\lambda$ are compatible. Thus $\cat{S}$ is both a symmetric monoidal $\infty$-category and a symmetric $\infty$-category over the symmetric monoidal category $\twovec$. 
\end{example}

\begin{remark}
\label{rem.twovect}
If $\twovec$ is a symmetric monoid
\end{document}
al category, then we have $X \times Y = [X;Y]$ by tensoring $X$ with $Y$ and recalling that multiplication is given by the tensor product of $X$ and $Y$. This definition is equivalent to the usual formulation of tensor product. That is, $X \times Y = X \otimes Y$. To simplify notation, we define $X \otimes Y = X \times Y$. The symmetric $\infty$-category $\cat{S}$ is just the symmetric monoidal $\infty$-category with the same objects as $\cat{S}$ but for an endomorphism $X \rightarrow Y$ in $\cat{S}$ with co-adjoints $\kappa$ and $\lambda$, such that $\cat{K}': X \rightarrow Z$ is the image of $\cat{K} = \kappa \otimes \lambda$ in $\twovec$. We define the symmetric $\infty$-category $\CAT{D}^H$ associated to $\twovec$ as the homotopy category of $\cat{S}$. 
\end{remark}

\section{Construction Of Higher Categories From Other Categories}

This section is a general overview of