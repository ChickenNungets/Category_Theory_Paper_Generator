
\documentclass[a4paper,reqno,oneside]{article}
\pdfoutput=1
\include{mathcommands.extratex}
\begin{document}
\title{On The Monoindal Endofunctor In Sparse Categories}
\author{Max Vazquez}
\maketitle


We introduce our first non-strict symmetric monoidal endofunctor in the case of sparse categories known as \textit{toposes}, to be understood in context with $\sset$-enriched categories or topological spaces (see e.g.\~\cite{Jacobs2016}). The functor admits a canonical description using its coend of dual. We also construct our notion of subtleties in the context of toposes and find that in many cases we can formulate a new definition of endofunctors for toposes (see e.g.\~\cite{Bourke2023} or \cite{Savage2017,Bourke2022}). 

In this paper, we provide several constructions of topos-enrichment that are relevant to our main result in this paper (see Appendix A). We introduce:
\begin{itemize}
    \item The category $\scat$ with finite products and coproducts, called the \textit{category of toposes}. The category has $\infty$-morphisms, i.e.\ the arrows from $I_n\to I_{n+1}$, where $I_0=\varnothing$, and the coends of these morphisms form a full subcategory (i.e.\ is a topos).
    \item The category of small $K$-toposes which is generated by a small set $T_K$. Moreover, given an open cover of $T_K$, it belongs to the full subcategory of the topos corresponding to its closed cover.
    \item The category $\sset$ whose objects are small sets and whose morphisms are pairs $(A,\alpha)\to (B,\beta)$ of sets. 
    \item The monoidal category of sheaves on a topos, i.e.\ a \textit{topological sheaf category}. These are obtained via the Yoneda lemma from \cite[Section 5.8]{Rozenblyk2003}.
\end{itemize}
Using these categorical structures, we proceed to establish an explicit statement about our notions of toposes and subtleties. This will be used as a guide in applying our results to toposes to prove a main result in \cite{Jacobs2016} (see Theorem 5). 



The authors have been referred to the following publications for additional background on toposes:
\begin{enumerate}
    \item \textbf{A model structure on toposes by Juan Garcia Barrack and Nicolas Jacobson}: \href{https://arxiv.org/abs/1904.06288}{\tt arXiv:1904.06288} \\
    \textbf{Representable Monoidal Categories in Representable Toposes}. \href{https://arxiv.org/abs/2010.12084}{\tt arXiv:2010.12084}

    \item \textbf{Towards a Universal Reconstruction of Towards Reproducible Science: From Prospective Researchers to Professionals}: \href{https://arxiv.org/abs/2010.08400}{\tt arXiv:2010.08400}

\end{enumerate}

For a general introduction to toposes see e.g.\~\cite{DubucReitberg2012,Johnstone2008}.


The abstract below gives some background about toposes. Please note that the definitions are taken from the literature and in no way represent those of the author. We only give important background. It should be noted that while they may have been considered before their name, there is no ambiguity regarding whether an object in a topos has ``dual" or ``codomain". Here we use the same term as in \cite{DubucReitberg2012}, because although the terminology is slightly confusing, our approach uses the latter notion as well.

\subsubsection*{General Background}

Let $\mathbf{X}$ denote a set. An \textit{object} in $\mathbf{X}$ consists of an element $x$ together with two additional data of a pair $(i,\mu)$. When $x$ and $y$ are both real numbers, $i$ may be anything and $j$ will be a function taking values between $0$ and $1$. Then $(i,\mu)$ can be thought of as a function which maps any other element $y$ to the value $i(y)$. For example, if $x$ and $y$ are both real numbers and $i$ takes values between $0$ and $1$, then $(x,i)=(0,i)$ and $(y,i)=(1,1)$. In this notation, we say that $x$ is called an \textit{input} and that $y$ is called an \textit{output}, in the sense that $x$ gets its output from $y$. In particular, any set contains all inputs and outputs. A \textit{map} of $\mathbf{X}$ is a function $f:\mathbf{X}\to\mathbf{X}$. An \textit{$i$-ary map} of $\mathbf{X}$ is a function $f:\mathbf{X}^{i}\to\mathbf{X}$. An \textit{operation} is an operation symbol $\mu$ followed by an $i$-ary map $\mu_i$. The type of such an operation (for instance, $\mu_{0}:\mathbf{X}^{0}\to\mathbf{X}$ or $\mu_0:\mathbf{X}^{0}\to\mathbf{X}$) is the type of $\mathbf{X}^{i}$. For example, the type of a map $f:\mathbf{X}\to\mathbf{Y}$ is just $(\mathbf{X}^{0},f):\mathbf{X}\to\mathbf{Y}$; the type of a map of objects is simply $(\mathbf{X}^{\times},f):\mathbf{X}\to\mathbf{Y}$. For two types of operations $f_1:\mathbf{X}^{0}\to\mathbf{X}$ and $f_2:\mathbf{Y}\to\mathbf{Z}$, there is a unique type of operation $(f_1,f_2):(\mathbf{X}^{0},f_1)\to(\mathbf{Y},f_2)$. For a pair of functions $g:(f_1,\mu_1)\to (f_2,\mu_2)$ and $h:(f_1',\mu'_1)\to (f_2',\mu'_2)$, there is a unique type of operation $(g,h):\begin{bmatrix} (\mathbf{X}^{0},f_1) \\(\mathbf{Y},f_2) \end{bmatrix}\to(\begin{bmatrix} (\mathbf{X}^{0},f_1') \\(\mathbf{Y},f_2')\end{bmatrix})$. One can view $\mathbf{X}$ as a finite product of sets of finite products and maps between them. It is well known that this sets correspond uniquely to the sets of inputs and outputs of an operation.

If $\mathbf{X}$ is an object in $\mathbf{V}$ then we may consider the free product $\mathbf{V}\times_{\mathbf{X}}_{\mathbf{X}}$ over $\mathbf{X}$ as the object in $\mathbf{X}$ satisfying the map $(x,\mu)\mapsto (x,\mu(x))$. If $\mathbf{X}$ is a map $f:(x,\mu)\to (y,\nu)$, then it has a corresponding product map $f^{\mathbf{X}}:(x^{\mathbf{X}},\mu^{\mathbf{X}}\to (y^{\mathbf{X}},\nu^{\mathbf{X}})$ satisfying the following equations:
\[
\mu^{\mathbf{X}}\circ f^{\mathbf{X}}(x,\mu(x)) = y^{\mathbf{X}}\circ f^{\mathbf{X}}(y,y),\qquad \mu^{\mathbf{X}}\circ f^{\mathbf{X}}(x,y) = \nu^{\mathbf{X}}\circ f^{\mathbf{X}}(x,x)
\]
Then we can define:
\[
f^{\mathbf{X}}=(x,\mu)\mapsto x^{\mathbf{X}},\qquad g^{\mathbf{X}}=(y,\nu)\mapsto y^{\mathbf{X}}.
\]
Furthermore, when $\mathbf{X}$ is a map and $x$ and $y$ are both real numbers, we write $(x,\mu)\cdot (y,\nu) = (x\cdot y, \mu\cdot \nu)$ as the composite of $\mu$ and $\nu$
and similarly $(x,\mu)(y,\nu) = (x\cdot y,\mu(y)\cdot \nu)$. The set $\mathbf{X}^{0}$ has the type $(x,\mu):\mathbf{X}^0\to\mathbf{X}$ and its domain is the set $\{x\}$. This type of set can be viewed as a product of sets. Similarly, the set $\mathbf{X}^1$ has the type $(x,\mu):\mathbf{X}^1\to\mathbf{X}$ and its domain is the set $\{x,y\}$. This type of set can be viewed as a product of sets. Furthermore, the type of a map $f:(x,\mu)\to (y,\nu)$ can be viewed as a combination of the types of $f^{\mathbf{X}}$ and $g^{\mathbf{X}}$. Thus, if $x$ and $y$ are input elements of an operation then we write $(x,\mu)\cdot (y,\nu)$ as the composite of $x$ and $y$ together with $\mu$ and $\nu$. Conversely, if $(x,\mu)$, $(y,\nu)$ and $(z,\lambda)$ are maps of objects then we write $(x,\mu)\cdot (y,\nu)\cdot (z,\lambda)=x\cdot (y\cdot z,\lambda)$ as the composite of $(x,\mu)$ and $(y,\nu)$ together with $(z,\lambda)$.  

Consider the product $\mathbb{N}_{\leq0}^\oplus$ of sets of cardinality at most zero. Denote the type of an object in $\mathbb{N}_{\leq0}^\oplus$ by $\mathsf{Top}$. A map of $\mathbb{N}_{\leq0}^\oplus$ is defined as $(x,\mu):\mathbb{N}_{\leq0}^\oplus\to\mathsf{Top}$, where the function $\mathsf{Top}(x,y):=(x,y)$ sends a natural number $x$ to the type $\mathsf{Top}(x,\mu):=(x,\mu)$ and an operation symbol $\mu$ to the type $(x,\mu):\mathbf{X}^0\to\mathbf{X}$. A map of $\mathbb{N}_{\leq0}^\oplus$ is called a \textit{morphism}. If $\mathsf{Top}$ is an $\infty$-topos, there is an $\infty$-topos called the \textit{$\mathbb{N}_{\leq0}^\oplus$-topos} generated by the map of $\mathbb{N}_{\leq0}^\oplus$ and the set of morphisms.  A $\mathbb{N}_{\leq0}^\oplus$-topos contains a single object and a single set of morphisms $X$, thus it can be seen as a category. 

A \textit{closed subset} of an object in an $\infty$-topos is a family of maps $[f]=[f_0]...[f_n]$, where each $f_i:X\to Y_i$ is an inclusion map of $\mathbb{N}_{\leq0}^\oplus$. If $X$ has all closure under composition, then the type $(x,\mu):\mathbf{X}^0\to\mathbf{X}$ is equivalent to $x$ being in the closure of $f^{\mathbf{X}}$ (that is, a collection of $f_0,...,f_n$). Then we denote the full subcategory spanned by closed subsets of objects by $\mathbf{X}^\ell$ and $[f]$ is a closed subset of $(x,\mu):X^\ell\to \mathsf{Top}$ iff $f^{\mathbf{X}}$ is an inclusion map for every element $x$ of $X$. The type $\mathsf{Top}$ is closed under composition.

The category $\mathbf{V}$ can be viewed as a product of finite products in an $\infty$-topos. In particular, objects in $\mathbf{V}$ have two types: 
\begin{itemize}
    \item the set $\mathbf{V}^0$, and its domain is $X$. This type of set can be viewed as a product of sets $X^{\bullet}$ or $X^{\ast}$.
    \item the set $\mathbf{V}^1$, and its domain is the set $X^{\ast}$ (i.e.\ the point).
\end{itemize}
There is one set $\mathbf{V}^0$ with domain $X$ and $\mathbf{V}^1$ with domain $X^{\ast}$. Let $\mathbf{V}^i$ denote the product $\mathbf{V}^i$ over $X$, where $i\geq 0$. There is a family of types of $X$ equipped with relations $(x_0,x_1)\sim (y_0,y_1)$ iff $x_0=y_0$ and $x_1\leq y_1$ (resp., $y_0=x_1$ and $x_0\geq y_1$) in $X$ (see Remark 1 below). Since $(x,\mu):\mathbf{X}^0\to\mathbf{X}$ is a map of $\mathbf{V}^i$ and the types $x_0$ and $x_1$ above are equipped with relatives $(x_0,x_1)\sim (y_0,y_1)$, then there is an order preserving relation $(x_0,x_1)\leq (y_0,y_1)$ on $X$ satisfying $x_0\leq y_0$ and $x_1\leq y_1$. Similarly, the family of types $(y_0,y_1)$ becomes $(x_0,\mu)$ and $(x_1,\nu)$ whenever $y_0=x_1$. So $Y=(y_0,y_1)\in X$ is an inclusion map if and only if $[y_0,y_1]\subseteq [x_0,x_1]$.

A morphism in $\mathbf{V}^i$ is a map $[f]:X^{i}\to Y_i$ of $\mathbf{V}^i$, which is a $\mathbf{V}^i$-morphism iff $f:X^{(i)}\to Y^{(i)}$ is an inclusion of $\mathbf{V}$ with respect to this relation. To see how these are related, let $(\mathbf{V}^i,\mu_0):X^{i}\to Y_i$ and $(\mathbf{V}^j,\mu_1):Y^{j}\to Z_{i,j}$ be two elements in $\mathbf{V}^i$ and suppose $(y_0,y_1)\sim (z_0,z_1)$ iff $\mu_0(y_0)=\mu_1(y_1)$ and $y_0\leq z_0$ and $y_1\leq z_1$. Then there exists a relation $(y_0,y_1)\sim (z_0,z_1)$ in $X$ such that $y_0=z_0$ and $y_1=z_1$.

Note that any element of $X$ is always represented by either $X^0$ or $X^1$. Let $(x_0,x_1)\in X^i$ be an element of $X$. Then $(x_0,x_1)$ and $x_0=y_0$ and $x_1=y_1$ in $X$ iff $y_0=y_1$. The relation $X^i\sim X^j$ is said to be compatible iff there exists a map $(\mu_0, \mu_1)$ making $(x_0,x_1)\sim (y_0,y_1)$ iff $y_0=y_1$. Note that if $(x_0,x_1)\not\sim (y_0,y_1)$, then $(x_0,x_1)\not\sim (z_0,z_1)$ so $y_0=z_0$ and $y_1=z_1$.

To see that every element of $X$ can be represented by either $X^0$ or $X^1$, we let $x_0$ and $x_1$ be arbitrary elements of $X$. Suppose that $(x_0,x_1)\sim (y_0,y_1)$ is compatible with $(y_0,y_1)$. Then we can find a map $[x_0]:X^{i}\to Y_i$ which is an inclusion of $\mathbf{V}^i$ with respect to the compatibility condition of $(x_0,x_1)$. If $y_0=y_1$, then $[x_0]$ is an inclusion of $X^{i}$ with respect to compatible conditions. Otherwise, if $y_0\leq z_0$ and $y_1\leq z_1$ then $[x_0]$ is an inclusion of $X^{i}$ with respect to compatible conditions. If $y_0>z_0$ and $y_1>z_1$, then $[x_0]$ is an inclusion of $X^{i}$ with respect to incompatible conditions. Similarly, we find a map $[x_1]:X^{i}\to Y_i$ which is an inclusion of $\mathbf{V}^i$ with respect to incompatible conditions. Finally, if $y_0<z_0$ and $y_1<z_1$, then $[x_1]$ is an inclusion of $X^{i}$ with respect to incompatible conditions. 


\subsubsection*{Relevant Literature}

A topos is a closed category with reflexive relations. For any two types $x,y$ of objects in $\mathbf{V}$ such that $x\sim y$, there exists a family of types of $X$ and relations $(x_0,x_1)\sim (y_0,y_1)$ such that $x_0=y_0$ and $x_1=y_1$ (resp., $y_0=x_1$ and $y_1=x_0$). An object in $\mathbf{V}$ is called a \textit{regular product} (or \textit{open set}) when it is an open set equipped with a regular product of types of $X$ and $(x_0,x_1)\sim (y_0,y_1)$. The set of open sets is closed under the regular products, but a morphism in an $\infty$-topos is called an \textit{open map} (or \textit{regular map}) if it is a regular map of $X$. A regular map in $\mathbf{V}$ is called an \textit{open embedding}. A morphism in $\mathbf{V}$ is called an \textit{open homomorphism} if it is a regular map of types $Y$. An object of $\mathbf{V}$ is called \textit{closed} if its closure under composition coincides with its domain.

If $\mathbf{X}$ and $\mathbf{Y}$ are toposes and $x$ and $y$ are objects, we call $x$ and $y$ an \textit{equivalence class} of $x$ and $y$ respectively. 

An object in $\mathbf{X}$ is called \textit{normal} if $\mathbf{X}^0\subseteq X$ is a regular product and $x\sim y$ holds in the equivalence class of $x$ and $y$. A morphism of $\mathbf{X}$ is called \textit{normal} if $f:X^\ell\to Y^\ell$ is a regular map.

In the theory of toposes, if an arrow $f:x\to y$ is a normal morphism, then $f^{-1}(x)$ and $f^{-1}(y)$ are isomorphic (the latter being the image of $f$). By definition, $\mathbf{X}^0$ is the point. 

If $\mathbf{X}$ is an $\infty$-topos, then the type $(x,\mu):X^0\to\mathbf{X}$ is called a \textit{sheaf} and a regular map $[f]:X^0\to Y$ is called a \textit{sheaf map} if it is an open homomorphism.

When $\mathbf{X}$ is a sheaf category and $x$ and $y$ are elements in $\mathbf{X}$, the types $(x,\mu):X^0\to\mathbf{X}$ and $(y,\nu):Y^0\to\mathbf{Y}$ are called a \textit{homotopy pair}, respectively. Sheaf categories are \textit{linear} when the domain of $\mathbf{X}$ is isomorphic to $\mathbf{X}^1$ and the type $(x,\mu):X^0\to\mathbf{X}$ is a \textit{linear sheaf}. Sheaf categories are \textit{symmetric} when $\mathbf{X}$ is a subsheaf category and the types $(x,\mu):X^0\to\mathbf{X}$ and $(y,\nu):Y^0\to\mathbf{Y}$ are homotopy pairs. If $x$ and $y$ are both elements in $\mathbf{X}$, then there is a unique map $(x,\mu):X^0\to\mathbf{X}$ and $(y,\nu):Y^0\to\mathbf{Y}$ such that $x\sim y$.

A \textit{small $\mathbb{N}_{\leq0}$-topos} is a small $\mathbb{N}_{\leq0}$-topos equipped with an ordinary topology. Let $\mathbf{P}$ denote the category of small $\mathbb{N}_{\leq0}$-toposes. In fact, $\mathbf{P}$ is a topos and every small $\mathbb{N}_{\leq0}$-topos is a topos. 

A category of toposes is called a topos-enrichment of a category $\mathbf{X}$ and a set $T$ of small $\mathbb{N}_{\leq0}$-toposes is called a \textit{small $\mathbb{N}_{\leq0}$-topos enrichment} of a category $\mathbf{X}$. For a small $\mathbb{N}_{\leq0}$-topos, we define a new category $\mathbf{X}\hookrightarrow \mathbf{P}$ to be a full subcategory of $\mathbf{P}$ consisting of small $\mathbb{N}_{\leq0}$-toposes whose basepoint object is equal to the empty set and whose basepoint morphism is an arbitrary map. For a small $\mathbb{N}_{\leq0}$-topos $X$, the category of small $\mathbb{N}_{\leq0}$-topos enriched with $X$ is called the \textit{category of small $\mathbb{N}_{\leq0}$-topos enrichments of $X$} and denoted by $\mathbf{X}\leftarrow X$. For a small $\mathbb{N}_{\leq0}$-topos $X$ and a small $\mathbb{N}_{\leq0}$-topos $Y$, a morphism $f:X\to Y$ is called a \textit{$\mathbb{N}_{\leq0}$-topos embedding} if $f:\mathbf{X}^1\hookrightarrow \mathbf{P}$ is an embedding and the composition map $[\bullet,\bullet]:X^2\to \mathbf{P}$ is an isomorphism. 

For a small $\mathbb{N}_{\leq0}$-topos $X$, we can define a \textit{$\mathbb{N}_{\leq0}$-topos embedding} $f:X\hookrightarrow X$ by setting $f(x,x)=1$ whenever $x\in X$ and 0 otherwise. A morphism of $\mathbb{N}_{\leq0}$-toposes $f:X\hookrightarrow Y$ is called a \textit{$\mathbb{N}_{\leq0}$-topos surjection} if $f(x)=1$ for all $x\in X$ and 0 otherwise.


The category of $\mathbb{N}_{\leq0}$-toposes enriched with the empty set as the basepoint object (recall that the basepoint map is the identity) is called the category $\mathbf{P}^{empty}$. Then $\mathbf{P}^{empty}$ is the category $\mathbf{P}$ of small $\mathbb{N}_{\leq0}$-toposes. 

By Proposition 2.2 in \cite{DubucReitberg2012}, the $\mathbb{N}_{\leq0}$-topos category of small $\mathbb{N}_{\leq0}$-toposes is equivalent to the category $\mathbf{P}^{0}$ of $\mathbb{N}_{\leq0}$-toposes equipped with an ordinary topology, that is, the set of regular products of $X^0$ for every $X$. A functor $F:X\to Y$ between toposes is called a \textit{surjection} between $\mathbb{N}_{\leq0}$-toposes if $F(x)=1$ for all $x\in X$. A morphism of $\mathbb{N}_{\leq0}$-toposes $f:X\hookrightarrow Y$ is called a \textit{surjection} if $f(x)=1$ for all $x\in X$. The category of $\mathbb{N}_{\leq0}$-toposes enriched with the point as the basepoint object is called the category of toposes.

The category $\mathbf{P}$ is known as a $\mathbb{N}_{\leq0}$-topos category since it is a full subcategory of $\mathbf{P}^{0}$. The category of $\mathbb{N}_{\leq0}$-toposes enriched with itself is called the category of \textit{toposes} and denoted by $\scat$ or $\sset$ for short. 

By Corollary 3.4 in \cite{Savage2017}, a small topos is equivalent to a sheaf category with a homotopy coherent category of finite products. Thus, any small topos $X$ is equivalent to a finite product of sheaves on $X$ by construction. 

The category of toposes enriched with itself is called the \textit{sheaf category} and denoted by $\scat$ or $\sheaf$. As shown in \cite{Savage2017}, the category of sheaves on a topos is equivalent to the category of sheaves over a topological space, hence to a finite product of sheaves on a topological space. 

A topological space is called a topos if the category $\sset$ of small sets has finite products. 

A sheaf category is called a topos-enrichment of a sheaf category $\sheaf$ and $T$ is called a sheaf $\sheaf$-topology if $\sheaf$ has a sheaf topology and every normal map is a sheaf topology morphism. A sheaf category $\sheaf$ is called a topos-enrichment of $\sheaf$ and $T$ is called a topos-topology if $\sheaf$ has a topos topology and every normal map is a topos-topology morphism. A sheaf category is called a \textit{topos-enrichment} of $\sheaf$ and $T$ is called a \textit{sheaf topology} if $T$ is a topos-topology.  

As explained above, a sheaf category enriched with itself (as opposed to the sheaf category itself) is called a \textit{sheaf category with homotopy coherent finite products} and denoted by $\scat\fint$ or $\sset\fint$. 

The category $\scat$ can be viewed as a finite product of sheaves and toposes. Let $\scat_{X,Y}$ be the category $\scat$ equipped with the product map $X\times Y\to X$, where the basepoint object is the empty set and each type of a sheaf is a normal map from $X$ to $Y$. Let $\sheaf_{X,Y}$ be the category of sheaves on $\scat_{X,Y}$ and sheaf $\sheaf_{X,Y}$-topology is the topologies defined by $\sheaf_{X,Y}^0=\sset$, $\sheaf_{X,Y}^1=\sset$ and $\sheaf_{X,Y}^2=\sset$. The category of sheaves on $\scat_{X,Y}$ is called the \textit{sheaf category on $X$ and $Y$}. 

A \textit{large topos} is a small $\mathbb{N}_{\leq0}$-topos with an ordinary topology. This includes all the topos-enrichment theorems for sheaf categories.

An $\infty$-category has a weak limit if there exist strong $\infty$-limits $L_0,L_1,...L_\infty$ on the underlying set that are preserved by the basepoint morphism. 

A $2$-categorical Grothendieck topos $\mathbf{W}$ has weak limits if every object $X$ has a weak limit in $\mathbf{W}$. A Grothendieck topos $\mathbf{W}$ is said to have weak colimits if every normal map in $\mathbf{W}$ has a weak colimit and every object in $\mathbf{W}$ has a weak colimit in $\mathbf{W}$. A Grothendieck topos $\mathbf{W}$ is said to have regular limits if every normal map in $\mathbf{W}$ has a regular limit.

A functor $F:\mathbf{X}\to\mathbf{Y}$ is said to have a left adjoint if $F$ has a right adjoint and the adjunction satisfies a standard condition. An object $X$ in a $2$-category $\mathbf{C}$ is said to have a left adjoint $L_X:\mathbf{X}\rightrightarrows \mathbf{C}$, an object $X$ in a $2$-category $\mathbf{C}$ is said to have a right adjoint $R_X:\mathbf{C}\rightleftarrows \mathbf{X}$ and the adjunction satisfies an adjoint-lifting condition.

%There is a forgetful functor $U:\mathbf{C}\to\mathbf{W}$ from the category of left adjoints in $\mathbf{C}$ to the category of right adjoints in $\mathbf{C}$ and a forgetful functor $U:\mathbf{C}\to\mathbf{C}$ between $2$-categories. The forgetful functors are fully faithful. 

An enriched category $\mathbf{C}$ is called a $2$-topos if it has weak limits. A $2$-topos is called a $2$-topos-enrichment of a $2$-category $\mathbf{C}$ and every normal map is a $2$-topos-enrichment morphism. 

\subsubsection*{Definition}

\begin{definition}[Topos]
A \textit{topos} is a small $\mathbb{N}_{\leq0}$-topos equipped with an ordinary topology. 
\end{definition}

\begin{remark}
A topos is equivalent to a large topos.
\end{remark}

\begin{definition}[Sheaf Category]\label{def:SheafCat}
Let $\sheaf_{X,Y}$ be a category whose basepoint object is a singleton set $X$ and whose objects are pairs of $X$-elements $x$ and $y$ such that $x\sim y$. A \textit{sheaf category} is a category $\sheaf_{X,Y}$ equipped with a sheaf topology $\sheaf_{X,Y}^0$ and an $X$-indexed category $\sheaf_{X,Y}^1$. A $2$-topos-enrichment $\sheaf_{X,Y}$ is a sheaf category if it has a sheaf topology and if every normal map is a sheaf topology morphism. A sheaf category is a $2$-topos-enrichment if it has a sheaf topology. A $2$-topos-enrichment $\sheaf_{X,Y}$ is called a sheaf category with homotopy coherent finite products if it has weak colimits. 
\end{definition}



\subsection*{Main Result}

For a small $\mathbb{N}_{\leq0}$-topos $\mathbf{X}$, we shall now formulate the main results in this paper. We shall first formulate a theorem about the coend of a topos and show that this is equivalent to the topos-enrichment of a sheaf category. Next
\end{document}
, we will construct a non-strict symmetric monoidal endofunctor for $\mathbf{X}$, showing that the two result are equivalent. Lastly, we show that we can formulate a notion of subtleties in the context of toposes and find that in many cases we can formulate a new definition of endofunctors for toposes (see e.g.\~\cite{Bourke2023} or \cite{Savage2017,Bourke2022}). 

We provide several constructions of topos-enrichment that are relevant to our main result in this paper (see Appendix A). We introduce:
\begin{itemize}
    \item The category $\scat$ with finite products and coproducts, called the \textit{category of toposes}. The category has $\infty$-morphisms, i.e.\ the arrows from $I_n\to I_{n+1}$, where $I_0=\varnothing$, and the coends of these morphisms form a full subcategory (i.e.\ is a topos). 
    \item The category of small $K$-toposes which is generated by a small