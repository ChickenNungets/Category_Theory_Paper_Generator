
\documentclass[a4paper,reqno,oneside]{article}
\pdfoutput=1
\include{mathcommands.extratex}
\begin{document}
\title{Co-Algebras: Why Do They Exist?}
\author{Max Vazquez}
\maketitle


\begin{abstract}
    In this paper, we introduce a notion of co-algebra for the category of modules over a ring $\R$.  We then show that co-algebras exist for a variety of abelian groups with characteristic $n$ (i.e., the number of isometries), and we prove in fact that there are exactly two classes of co-algebras on the category of modules over $\Z/2^n$.  The second property relies upon our use of cohomology to characterize the $n$-dimensional tensor product.  Our results are also general enough to capture all nontrivial algebras over any ring that admit characteristic zero.  Throughout the paper, we review some related ideas in algebraic geometry, including the classical examples of cohomology group, Hilbert space, and toric series.
\end{abstract}

\setcounter{tocdepth}{2}
\tableofcontents
\section{Introduction}
An object $(X,\tau)$ in an operad $\mathcal{P}$ is called \emph{$\mathcal{P}$-comonoid}, which consists of a fixed set of objects $A$ and a fixed collection of morphisms from each of those $A$ to itself.  A $\mathcal{P}$-module is an object in $\mathcal{P}$ equipped with a structure map $\sigma : A \to X$ such that every comultiplication $\tau_\sigma : X \otimes A \to A$ preserves the structure map.  This means that $\sigma$ preserves coproducts, products, and pullbacks.  The associated comonad $(-,\mu) : X \to \mathcal{P}$, i.e., a map $m : A \to X$, gives rise to a \emph{monad $\mu : A \to X$} (see \cite[Chapter III.5]{BourkeBook}) whose unit $\eta$ has codomain $\mu(A) = A$ and counit $\epsilon : \mu^{-1}(A) = A \otimes -$.  Cocomonoids are often used in the study of category theory as the monads which will be studied below.  

\begin{example} \label{example:cooperads}
 Let $\V$ be a topological vector space.  There is a comonoid structure $\vminus : \V^\times \to \V^\times$, where each component has the form $[x]_a = x$, the vector space consisting of elements $x$ and components (or edges) that have no direction except up or down, while $[-]_{b} = [b]$ represents the comonoid homomorphism $- \colon \V^\times \to \V^\times$ from the set $\V$ to itself, but which maps all directions to $0$.  If $u, v \in \V^\times$, then the following diagram commute:
 \[
 \begin{tikzcd}[row sep=3pc, column sep=1.8pc]
  [u] \ar[d,"\lambda"] \ar[r,"\alpha_u"] & [v] \ar[d,"{\lambda}"'] \\
  \V \ar[r,"\lambda"']               & \V \ar[d,"{\lambda}"]
 \end{tikzcd}
 \]
We call $\vminus$ the \emph{cooperad}.  Given a morphism $f : A \to B$ in $\V^\times$, the \emph{coproduct of $f$ with $g$} is its image under the comultiplication $\sum_{a} f_a + g_a$, which is isomorphic to $[u] + [v] = f + g$.  If $f : A \to B$ and $f' : B \to C$ are $\V$-modules, then the \emph{pushforward of $f$ along $f'$} is given by
\[
[x] = [f] + \left(\sum_{a} f_a\right) + \left(\sum_{b} g_b\right).
\]  
The operation $+$ is given by the usual sum over commutative diagrams and the composition of morphisms is given by taking the pushforward of the leftmost morphism (i.e., the rightmost diagram on the left) along the rightmost morphism (the rightmost diagram on the right).
\end{example}


Let us note that a comonoid $\mu : A \to X$ is \emph{strongly unital} if $- \mu$ is an invertible comultiplication (i.e., every morphism factors uniquely through the identity).  A comonoid $\mu$ is \emph{additive} if it preserves injective maps.  In fact, additivity was already well established in \cite{GarnerLack}, and is not restricted to coalgebras; in fact, there is nothing restricting a comonoid to being coadditive without concessions.  Strong unital comonoids are always coadditive as well.  For example, any finite-dimensional field can be written as a cocomonoid $\mathbb{F}_q$, where $q > 0$ is any prime number, and $\mathbb{F}_q(A,X) = \mathbb{F}_q(A,- \mu(A))$ for all $A,X \in \mathbb{F}_q$.  The \emph{additive symmetric group}, denoted $\mathbb{S}_k$, is generated by $k!$ and $\phi(m,n) = (-1)^{\sqrt{6}} m^{\gamma(n)} n^{-(k+1)}\left[k+1\right]^\frac{\gamma(n)}{\gamma(m)}\phi(m,n)$ for $0 < m < k$ and $0 < n < k+1$. 

\begin{example} \label{example:monoid-structure}
Let $\R$ be a real line.  Let $H_0 : \mathbb{C}^2 \to \mathbb{C}$ be the linear map $x, y \mapsto 0$ (resp.~$x^2 + y^2 = 0$) and let $E_0 : \mathbb{C}^2 \to \mathbb{C}$ be the identity map.  Then the cocomonoid $E_0 \times H_0$ has the structure given by
\[\left[x,y\right] = (-1)^{2\sqrt{3}}\big(\begin{pmatrix}x & -y \\ y & x\end{pmatrix}\right) = (-1)^{2\sqrt{3}}x+(-1)^{2\sqrt{3}}y = \left[0,0\right], \quad E_0 \times H_0 = 0.\]
Note that there is also another nonzero representation (resp.~the same one) of $E_0 \times H_0$ given by
\[\left[x,y\right] = (\sqrt{3})^{2} x y + (\sqrt{3})^{2} - y x = 0.\]
We also have the symmetric cocomonoid $E_0 \times E_0$, again with the structure given by
\[\left[x,y\right] = 2x - 2y, \quad E_0 \times E_0 = 0.\]
\end{example}


In addition to being coadditive, a comonoid $\mu : A \to X$ is said to be \emph{additive comonoidally} if $\mu$ preserves comultiplications and the comultiplication is commutative.  Comultiplication comonoidal categories are generally referred to as \emph{commutative categories} (\cite{GarnerLack}).  As we shall see in \cref{subsection:morphisms-comonoid-algebras}, there are many notions of co-algebra in this literature that come in different flavors, ranging from \emph{bimonoidal} coalgebras (i.e., \emph{two-sided} coalgebras); \emph{bimodule} coalgebras (i.e., \emph{dual} coalgebras); \emph{module} coalgebras (i.e., \emph{unit} coalgebras); \emph{algebras} (i.e., \emph{dualizable} coalgebras); etc.  These co-algebras arise in the study of linear groups and \emph{bialgebra} (i.e., algebras on modules) (see \cite{LurieTrottaBook}), but there is no formal treatment of them here since they do not necessarily exist in the literature.

The present paper aims to formalize the theory of co-algebras by providing an axiomatic definition of co-algebras and a characterization theorem theorems for various variants of co-algebras, including, but not limited to: \emph{(co)additive coalgebras}; \emph{(co)additive symmetric coalgebras}; \emph{co-module coalgebras}; \emph{co-module categories}; and \emph{co-module categories endowed with the monoid structure} (see \cref{section:co-algebras}).  In other words, it is an attempt to provide a reference point to understand co-algebras with the specific properties needed to describe certain coalgebraic structures such as the bilinear module category with dual \eqref{eq:module-category-with-dual} and the \emph{co-module category with monoid structure}; together with useful generalizations of coalgebraic structures.

The main goal of this paper is to establish a characterization theorem concerning co-algebras in terms of co-algebraic structures.  Here we focus our attention primarily on co-algebras on modules over an arbitrary ring.  However, we also expect the reader to know a few other facts about co-algebras at their disposal and to be able to apply our results to various categories in our next sections.


\section{Why Does Co-Algebra Exists?}
If we assume that our coalgebra is coadditive, then there are not only four relevant results concerning co-algebras (see \cite[Chap II.4]{BourkeBook}): \emph{counitality} (see \cref{definition:counitality}); \emph{equivariance} (see \cref{definition:equivariance)}; \emph{associativity} (see \cref{definition:associativity}); and \emph{unitality} (see \cref{definition:unitality}).  But what does it mean to have counitality?  That the co-unit $\epsilon$ has a trivial value in the category of modules over an arbitrary ring, is not unusual.  First of all, counitality is essentially the main reason why comonoids are important when working in algebraic topology: most of the time, comonoids are justified with unitality, in which case there is no need to work with co-algebras.  Secondly, counitality allows us to define coalgebras by adjoining counits on the free module over the ring $k$ (which we shall call the \emph{free co-module}).

% \begin{remark}
%     We should emphasize that $\epsilon$ acts transparently on the underlying category of the comonoid: in particular, $\epsilon(A) = A$.  This is because the action of $\epsilon$ on a comonoid is determined entirely by its underlying category.
% \end{remark}



Similarly, co-associativity is the main reason why comonoids are important when working in algebraic topology: most of the time, comonoids are justified with associativity, in which case there is no need to work with co-algebras.  Co-associativity allows us to define coalgebras by adjoining coassociators on the free co-module over $k$ (which we shall call the \emph{free co-module}).

Finally, co-unitality makes it possible to define co-comonoids and co-comodules.  As shown in \cref{subsection:co-comonoid}, these co-categories can be defined by adjoining cocomonoids and cocomodules to the free co-module over the ring $k$ (although this may change in future work).  And by showing that the co-comonoid associated to an object $A$ is the free co-comonoid with respect to the co-comonad structure on its domain $A$, we obtain an equivalent notion of co-comodule in the sense of \cite{BourkeBook}.

\begin{definition}
For a comonoid $\mu : A \to X$ and an object $A$ in $\V$, we define the \emph{co-unit} of $\mu$ as the corresponding comultiplication $\mu_A : \V \to X$ with the inverse $\epsilon_A : \V \to A$.
\end{definition}

\begin{lemma}\label{lem:counit-is-a-multiplication}
Let $\mu : A \to X$ and $\epsilon : A \to X$ be comonoidal comonads on $\V$.  Then there exists an element $t \in \V$ such that the comultiplication $\mu_A \circ \epsilon_A$ is isomorphic to the unit $\mu(A)$, for each $A \in \V$ (see \eqref{eq:counit-and-counit-on-a-comonoid}).
\end{lemma}
\begin{proof}
Note that there exists a unique $A \in \V$ satisfying the following diagram:
\[
\begin{tikzcd}[row sep=3pc, column sep=2pc]
\V \ar[rr, "u_A"] \ar[dr, "\mu_A"] &  & X \ar[dl, "\epsilon_A"] \\
& Y \ar[from=uu, to=dd, "t"']       &
\end{tikzcd}
\]
where $u_A : A \to \V$ is the action of $\epsilon$ on $A$ and $t$ sends $t_A : t \mu_A$ to $\epsilon(t_A)$, i.e.~$\mu(t_A) = \epsilon(t_A) = t$.
% \begin{center}
% \begin{tikzpicture}[xscale=.7]
% \filldraw[white,opacity=0.1] (-.8,-.7) rectangle (.8,.8);
% \foreach \x/\y/\z/\w in {0/2/1/0.5, 0/1/0/0.5, 1/1/2/1, 1/1/1/1.7, 1/2/1/1, 1/2/2/1.7}{
%     \draw ($ ($([0,1])_a/0.2-$x/1.6-$($x/1.6-z/1.6-w/1.6$)+0.2*$([1,0])/2$)!.5+$([1,1])_a$,$([1,1])_a/0.5+$($([0,1])_a/0.5-$x/1.6-$($x/1.6-z/1.6-w/1.6$)+0.2*$([0,1])/(1-y)+0.2*($([0,1])_a/2$)!)!.5+$([0,1])_a$)!.5$/$([1,0]/2$)!.5$/$([1,1]/2$)!.5$/$([1,2]/2$)!.5$/.
% }
% %
% \node at ($(0,0)$) {\scriptsize$A$};
% \node at ($(1,0)$) {\scriptsize$X$};
% \node at ($(2,0)$) {\scriptsize$Y$};
% \node at ($(0,1)$) {\scriptsize$\mathsf{Mod}_A$};
% \node at ($(1,1)$) {\scriptsize$\mathsf{Comod}_A$};
% \node at ($(2,1)$) {\scriptsize$\mathsf{End}_{\mathsf{mod}}$};
% \node at ($(0,2)$) {\scriptsize$\mathsf{Module}_A$};
% \node at ($(1,2)$) {\scriptsize$\mathsf{Comodule}_A$};
% \node at ($(2,2)$) {\scriptsize$\mathsf{End}_{\mathsf{comod}}$};
% \node at ($(0,-.7)$) {\scriptsize$A$};
% \node at ($(1,-.7)$) {\scriptsize$Y$};
% \node at ($(2,-.7)$) {\scriptsize$X$};
% \end{tikzpicture}
% \end{center}
Since $\epsilon_A : \V \to A$ is an epimorphism, there exists a unique $A \in \V$ such that the isomorphism $\mu_A \circ \epsilon_A$ is isomorphic to $\mu(A)$.  Since $\mu_A$ is isomorphic to $\epsilon_A$, this implies that $\mu(A) \in \V$.
\end{proof}

\begin{proposition}\label{prop:coalgebra}
The category of modules over a ring $\R$ has the following structure:
\begin{enumerate}[label=\roman*]
    \item Each object $(X,\tau)$ consists of an $\R$-module $X$ and a set of morphisms $\tau$.  
    \item Each morphism $(X,\tau) \to (Y,\theta)$ consists of an $\R$-module morphism $X \to Y$ and a set of morphisms $\theta$.
    \item Given two objects $(X,\tau),(Y,\theta)$, we have an equivalence of categories between
    \begin{equation*}
        \mathsf{Mod}_{X,Y}(\R) := \{f : X \to Y\mid \exists t \in \R, \textnormal{such that } f = \mu_A \circ \epsilon_A \textrm{ for } A \in X \}.
    \end{equation*}
\end{enumerate}
\end{proposition}
\begin{proof}
    By assumption, we have an equivalence of categories $\mathsf{Mod}_X(\R) \cong \mathsf{Mod}_Y(\R)$ between the categories of $\R$-modules and $\R$-modules and functors of modules (see \cite[Prop II.33]{BourkeBook} and \cite[Corollary 2.1]{GarnerLack}).  Furthermore, by Lemma \ref{lem:counit-is-a-multiplication}, there exists a unique element $t \in \R$ such that the comultiplication $\mu_A \circ \epsilon_A$ is isomorphic to the unit $\mu(A)$ for each $A \in X$ (see \eqref{eq:counit-and-counit-on-a-comonoid}).

    Now suppose that $f \in \mathsf{Mod}_{X,Y}(\R)$ is a natural transformation such that $\exists s \in \R$ such that the following diagram commutes:
    \[
\begin{tikzcd}[column sep=6pc]
\V^\times \ar[rr, shift right=1pt, "\epsilon_A"] \ar[rd, shift left=1pt, "u_A"] &  & \V \ar[d, "\mu_A"] \\
X^\times \ar[r, "f_X"] \ar[d, "f_X"]             & \V \ar[rr, "v"']             & X \ar[d, "v_X"] \\
Y^\times \ar[r, "f_Y"]                         & Y \ar[from=uu, to=dd, "t"']       &
\end{tikzcd}
\]
    Since each $A \in X$ corresponds to the free module over $k$ and thus the $A$-module $\V^\times$ is rigid, there exists a unique $A \in \V$ satisfying the following diagram:
    \[
    \begin{tikzcd}[row sep=3pc, column sep=2pc]
\V^\times \ar[rr, shift right=1pt, "\epsilon_A"] \ar[rd, shift left=1pt, "u_A"] \ar[dd, "\mu_A"] &  & \V \ar[d, "\mu_A"] \\
X^\times \ar[r, "f_X"]                       & \V \ar[rd, "\epsilon_A"]               & X \ar[r, "f_X"]                 & X^\times \ar[rd, "v"]               \\
Y^\times \ar[r, "f_Y"]                         & Y \ar[from=uu, to=dd, "t"]       &
\end{tikzcd}
    \]
    where $u_A : A \to \V$ is the action of $\epsilon$ on $A$ and $t : A \to Y$ sends $t_A : t \mu_A$ to $\epsilon(t_A) = f_Y(t)$.  Now suppose that $f_Y(t) \neq 0$.  We wish to find the unique $A \in \V$ such that $v_X \circ f_Y(t) = f_Y(s)$.  To do so, we compute the following diagram:
    \[
    \begin{tikzcd}[row sep=3pc, column sep=2pc]
        \V^\times \ar[rr, shift right=1pt, "\epsilon_A"] \ar[rd, shift left=1pt, "u_A"] &  & \V \ar[d, "\mu_A"] \ar[from=uu, to=dd, "t"'] \\
\V^\times \ar[rr, shift right=1pt, "\epsilon_A"] \ar[rd, shift left=1pt, "u_A"]     & \V \ar[rr, "v"']                         & X \ar[d, "v_X"] \ar[r, "f_X"]            & X^\times                              \\
Y^\times \ar[r, "f_Y"]                         & Y \ar[from=uu, to=dd, "t"]       &
\end{tikzcd}
    \]
    It is clear that $\epsilon_A \circ v_X = \epsilon_A \circ f_Y(t) = v(f_Y(s))$.  Thus there exists $A \in \V$ such that $v_X \circ f_Y(t) = f_Y(s)$.

    On the other hand, $v_Y \circ f_X$ is zero, so there is no unique $A \in \V$ such that $v_X \circ f_Y(t) = f_Y(s)$.
\end{proof}

The previous proposition establishes the structure of a coalgebra on a module.  Next we show how a coalgebra on a module gives rise to an algebroid on a module (see \cref{subsection:coalgebra-for-modules}).  While it might be interesting to study the comodules themselves, they do not appear to be sufficient for our purposes.  One could instead give an algebra on the co-comodule itself.

\begin{proposition}\label{prop:algebra-for-cocomodule}
Given a co-comodule $M$, we have the following characterization:
\begin{enumerate}[label=\arabic*]
    \item If $M$ is a direct summand of a comodule category $(X,\tau)$, then the associated comodule algebra $\mathsf{Alg}_{X,M}(\R) \simeq M$ is a direct summand of the comodule algebra $\mathsf{Alg}_X(\R) \simeq M$.
    \item Suppose that $M$ is a direct summand of a comodule category $(X,\tau)$ such that the respective comodule $\V^\times$ is closed.  Then the associated comodule algebra $\mathsf{Alg}_{X,M}(\R) \simeq M$ is the direct summand of the co-module algebra $(X,\tau)$ on itself.
\end{enumerate}
\end{proposition}
\begin{proof}
    (1) and (2) follow from \cref{prop:algebra-for-cocomodule} and \cref{prop:coalgebra} respectively.
\end{proof}

Here we conclude the paper by reviewing the relationship between coalgebras on modules and coalgebraic structures on comonoids.  Our main observation about coalgebras on modules follows from \cref{subsection:algebraic-structures-coalgebras}.  Our first application is illustrated in \cref{section:examples}.

\subsection{Examples}

There are many interesting applications of co-algebras.  In general, the most common of them are:
\begin{enumerate}
    \item (co)additive coalgebras;
    \item cocomodule coalgebras;
    \item co-module coalgebras.
\end{enumerate}
However, we do not discuss them here since coalgebras are not directly used here.  See \cref{section:more-examples} for more details.  For the record, we recall that co-comonoids arise from the theory of \emph{co-modules} as introduced by \cite{GarnerLack}, and that, in some sense, it was constructed as a direct sum of the algebras on a functor $M \to \R$ (see \cite[Def 2.3]{GarnerLack}).  Note that the category $\mathsf{Mod}_X(\R)$ arises from the functor $M : \R \to \mathsf{Vec}^*(k)$ (see \cite{BourkeBook}).  For the sake of brevity, we omit the index over $k$ whenever we consider a co-comodule over a commutative ring.  See Section \ref{subsection:coalgebra-as-a-functor-to-an-algebroid} below for further discussion.  Some examples of coalgebraic structures on comonoids are illustrated in \cref{example:cooperads,example:monoid-structure}.

% \subsubsection*{Modules}

First, we investigate the coalgebraic structures on co-comodules (co-comodules whose underlying co-module is a co-comodule).  Let us review the characterization of co-comodules on modules as described in \cref{subsection:coalgebra-as-a-functor-to-an-algebroid}.

\begin{proposition}\label{prop:coalgebra-on-modules-are-co-comodules}
Let $M$ be a comodule, and let $(X,\tau)$ be a cocomodule on $M$.  Then the following conditions are satisfied:
\begin{enumerate}[label=(\roman*)]
    \item The co-module $M$ has a canonical $X$-action.
    \item The co-comodule $M$ has a canonical $\tau$-action.
    \item The following comonoid structure diagram commutes:
    \begin{center}
        \begin{tikzpicture}[xscale=.7]
            \filldraw[blue!20] (-.2,-.5) rectangle (1.2,.5);
            \foreach \x/\y/\z/\w in {0/2/1/0.5, 0/1/0/0.5, 1/1/2/1, 1/1/1/1.7, 1/2/1/1, 1/2/2/1.7}{
                \draw ($ ([0,1])_a/0.2-$x/1.6-$($x/1.6-z/1.6-w/1.6$)+0.2*$([1,0])/2$)!.5+$([1,1])_a$,$([1,1])_a/0.5+$($([0,1])_a/0.5-$x/1.6-$($x/1.6-z/1.6-w/1.6$)+0.2*$([0,1])/(1-y)+0.2*($([0,1])_a/2$)!.5+$([0,1])_a$)!.5$/$([1,0]/2$)!.5$/$([1,1]/2$)!.5$/.
            }
            %
            \node at ($(0,0)$) {\scriptsize$X$};
            \node at ($(1,0)$) {\scriptsize$M$};
            \node at ($(2,0)$) {\scriptsize$N$};
            \node at ($(0,1)$) {\scriptsize$\mathsf{Mod}_M(\R)$};
            \node at ($(1,1)$) {\scriptsize$\mathsf{Comod}_M(\R)$};
            \node at ($(2,1)$) {\scriptsize$\mathsf{End}_{\mathsf{mod}}$};
            \node at ($(0,2)$) {\scriptsize$\mathsf{Module}_M(\R)$};
            \node at ($(1,2)$) {\scriptsize$\mathsf{Comodule}_M(\R)$};
            \node at ($(2,2)$) {\scriptsize$\mathsf{End}_{\mathsf{comod}}$};
        \end{tikzpicture}
    \end{center}
\end{enumerate}
\end{proposition}

\begin{proof}
The first condition immediately follows from the fact that both the free co-module and the underlying co-module of $M$ are co-comodules.  Therefore, (1) immediately follows from \cref{prop:coalgebra}.  To prove (2), let us denote by $\gamma : N \to M$ the morphism that sends $z \mapsto (-1)^{z/2}$ and $\psi : N \to N$ the morphism that sends $m \mapsto (1-z)/2$.  Since $(X,\tau)$ is a cocomodule, we can write $f = \epsilon_A + \sum_{a} (-1)^{a} \psi_A$ for an $\R$-module $A \in X$ and morphism $\sigma : A \to N$ such that $\tau_A$ is surjective and $\psi_A$ is injective.  Because $\mu_A$ is injective, there is a unique $A \in X$ satisfying the following diagram:
\[
\begin{tikzcd}[row sep=3pc, column sep=2pc]
N \ar[r, "f"] \ar[rd, "p"] &  & M \ar[d, "\mu_A"] \\
X \ar[r, "s"]         & M \ar[r, "f"]              & X \ar[r, "s"]                   &
\end{tikzcd}
\]
So, by construction, there exists a unique $A \in X$ such that $f = p + \sum_{a} (-1)^{a} \sigma_A$ is isomorphic to $s$.  Similarly, by assumption, there is a unique $N \in M$ such that $\tau(N) = f$.  So $(\mu_A, \epsilon_A, \tau_A, \psi_A)$ gives rise to an element $t \in \R$ such that the diagram commutes:
\begin{center}
\[
\begin{tikzcd}[column sep=6pc]
\V^\times \ar[rr, shift right=1pt, "\epsilon_A"] \ar[rd, shift left=1pt, "u_A"] \ar[dd, "\mu_A"] &  & \V \ar[d, "\mu_A"] \\
X^\times \ar[r, "f_X"]                       & \V \ar[rd, "\epsilon_A"]               & X \ar[r, "f_X"]                 & X^\times \ar[rd, "v"]               \\
Y^\times \ar[r, "f_Y"]                         & Y \ar[from=uu, to=dd, "t"]       &
\end{tikz
\end{document}
cd}
\]
\end{center}
Now, since $\mu_A$ is injective and $\epsilon_A$ is trivial, we can conclude by proving that the diagram commutes:
\begin{center}
\[
\begin{tikzcd}[column sep=6pc]
\V^\times \ar[rr, shift right=1pt, "\epsilon_A"] \ar[rd, shift left=1pt, "u_A"] \ar[dd, "\mu_A"] &  & \V \ar[d, "\mu_A"] \\
\V^\times \ar[rr, shift right=1pt, "\epsilon_A"] \ar[rd, shift left=1pt, "u_A"]     & \V \ar[rr, "v"']                         & X \ar[d, "v_X"] \ar[r, "f_X"]            & X^\times                              \\
Y^\times \ar[r, "f_Y"]                         & Y \ar[from=uu, to=dd, "t"]       &
\end{tikzcd}
\]
\end{